\documentclass[11pt]{article}
\usepackage{lipsum} % for dummy text
    \usepackage[breakable]{tcolorbox}
    \usepackage{parskip} % Stop auto-indenting (to mimic markdown behaviour)
    

    % Basic figure setup, for now with no caption control since it's done
    % automatically by Pandoc (which extracts ![](path) syntax from Markdown).
    \usepackage{graphicx}
    % Maintain compatibility with old templates. Remove in nbconvert 6.0
    \let\Oldincludegraphics\includegraphics
    % Ensure that by default, figures have no caption (until we provide a
    % proper Figure object with a Caption API and a way to capture that
    % in the conversion process - todo).
    \usepackage{caption}
    \DeclareCaptionFormat{nocaption}{}
    \captionsetup{format=nocaption,aboveskip=0pt,belowskip=0pt}

    \usepackage{float}
    \floatplacement{figure}{H} % forces figures to be placed at the correct location
    \usepackage{xcolor} % Allow colors to be defined
    \usepackage{enumerate} % Needed for markdown enumerations to work
    \usepackage{geometry} % Used to adjust the document margins
    \usepackage{amsmath} % Equations
    \usepackage{amssymb} % Equations
    \usepackage{textcomp} % defines textquotesingle
    % Hack from http://tex.stackexchange.com/a/47451/13684:
    \AtBeginDocument{%
        \def\PYZsq{\textquotesingle}% Upright quotes in Pygmentized code
    }
    \usepackage{upquote} % Upright quotes for verbatim code
    \usepackage{eurosym} % defines \euro

    \usepackage{iftex}
    \ifPDFTeX
        \usepackage[T1]{fontenc}
        \IfFileExists{alphabeta.sty}{
              \usepackage{alphabeta}
          }{
              \usepackage[mathletters]{ucs}
              \usepackage[utf8x]{inputenc}
          }
    \else
        \usepackage{fontspec}
        \usepackage{unicode-math}
    \fi

    \usepackage{fancyvrb} % verbatim replacement that allows latex
    \usepackage[Export]{adjustbox} % Used to constrain images to a maximum size
    \adjustboxset{max size={0.9\linewidth}{0.9\paperheight}}

    % The hyperref package gives us a pdf with properly built
    % internal navigation ('pdf bookmarks' for the table of contents,
    % internal cross-reference links, web links for URLs, etc.)
    \usepackage{hyperref}
    % The default LaTeX title has an obnoxious amount of whitespace. By default,
    % titling removes some of it. It also provides customization options.
    \usepackage{titling}
    \usepackage{longtable} % longtable support required by pandoc >1.10
    \usepackage{booktabs}  % table support for pandoc > 1.12.2
    \usepackage{array}     % table support for pandoc >= 2.11.3
    \usepackage{calc}      % table minipage width calculation for pandoc >= 2.11.1
    \usepackage[inline]{enumitem} % IRkernel/repr support (it uses the enumerate* environment)
    \usepackage[normalem]{ulem} % ulem is needed to support strikethroughs (\sout)
                                % normalem makes italics be italics, not underlines
    \usepackage{mathrsfs}
    

    
    % Colors for the hyperref package
    \definecolor{urlcolor}{rgb}{0,.145,.698}
    \definecolor{linkcolor}{rgb}{.71,0.21,0.01}
    \definecolor{citecolor}{rgb}{.12,.54,.11}

    % ANSI colors
    \definecolor{ansi-black}{HTML}{3E424D}
    \definecolor{ansi-black-intense}{HTML}{282C36}
    \definecolor{ansi-red}{HTML}{E75C58}
    \definecolor{ansi-red-intense}{HTML}{B22B31}
    \definecolor{ansi-green}{HTML}{00A250}
    \definecolor{ansi-green-intense}{HTML}{007427}
    \definecolor{ansi-yellow}{HTML}{DDB62B}
    \definecolor{ansi-yellow-intense}{HTML}{B27D12}
    \definecolor{ansi-blue}{HTML}{208FFB}
    \definecolor{ansi-blue-intense}{HTML}{0065CA}
    \definecolor{ansi-magenta}{HTML}{D160C4}
    \definecolor{ansi-magenta-intense}{HTML}{A03196}
    \definecolor{ansi-cyan}{HTML}{60C6C8}
    \definecolor{ansi-cyan-intense}{HTML}{258F8F}
    \definecolor{ansi-white}{HTML}{C5C1B4}
    \definecolor{ansi-white-intense}{HTML}{A1A6B2}
    \definecolor{ansi-default-inverse-fg}{HTML}{FFFFFF}
    \definecolor{ansi-default-inverse-bg}{HTML}{000000}

    % common color for the border for error outputs.
    \definecolor{outerrorbackground}{HTML}{FFDFDF}

    % commands and environments needed by pandoc snippets
    % extracted from the output of `pandoc -s`
    \providecommand{\tightlist}{%
      \setlength{\itemsep}{0pt}\setlength{\parskip}{0pt}}
    \DefineVerbatimEnvironment{Highlighting}{Verbatim}{commandchars=\\\{\}}
    % Add ',fontsize=\small' for more characters per line
    \newenvironment{Shaded}{}{}
    \newcommand{\KeywordTok}[1]{\textcolor[rgb]{0.00,0.44,0.13}{\textbf{{#1}}}}
    \newcommand{\DataTypeTok}[1]{\textcolor[rgb]{0.56,0.13,0.00}{{#1}}}
    \newcommand{\DecValTok}[1]{\textcolor[rgb]{0.25,0.63,0.44}{{#1}}}
    \newcommand{\BaseNTok}[1]{\textcolor[rgb]{0.25,0.63,0.44}{{#1}}}
    \newcommand{\FloatTok}[1]{\textcolor[rgb]{0.25,0.63,0.44}{{#1}}}
    \newcommand{\CharTok}[1]{\textcolor[rgb]{0.25,0.44,0.63}{{#1}}}
    \newcommand{\StringTok}[1]{\textcolor[rgb]{0.25,0.44,0.63}{{#1}}}
    \newcommand{\CommentTok}[1]{\textcolor[rgb]{0.38,0.63,0.69}{\textit{{#1}}}}
    \newcommand{\OtherTok}[1]{\textcolor[rgb]{0.00,0.44,0.13}{{#1}}}
    \newcommand{\AlertTok}[1]{\textcolor[rgb]{1.00,0.00,0.00}{\textbf{{#1}}}}
    \newcommand{\FunctionTok}[1]{\textcolor[rgb]{0.02,0.16,0.49}{{#1}}}
    \newcommand{\RegionMarkerTok}[1]{{#1}}
    \newcommand{\ErrorTok}[1]{\textcolor[rgb]{1.00,0.00,0.00}{\textbf{{#1}}}}
    \newcommand{\NormalTok}[1]{{#1}}
    
    % Additional commands for more recent versions of Pandoc
    \newcommand{\ConstantTok}[1]{\textcolor[rgb]{0.53,0.00,0.00}{{#1}}}
    \newcommand{\SpecialCharTok}[1]{\textcolor[rgb]{0.25,0.44,0.63}{{#1}}}
    \newcommand{\VerbatimStringTok}[1]{\textcolor[rgb]{0.25,0.44,0.63}{{#1}}}
    \newcommand{\SpecialStringTok}[1]{\textcolor[rgb]{0.73,0.40,0.53}{{#1}}}
    \newcommand{\ImportTok}[1]{{#1}}
    \newcommand{\DocumentationTok}[1]{\textcolor[rgb]{0.73,0.13,0.13}{\textit{{#1}}}}
    \newcommand{\AnnotationTok}[1]{\textcolor[rgb]{0.38,0.63,0.69}{\textbf{\textit{{#1}}}}}
    \newcommand{\CommentVarTok}[1]{\textcolor[rgb]{0.38,0.63,0.69}{\textbf{\textit{{#1}}}}}
    \newcommand{\VariableTok}[1]{\textcolor[rgb]{0.10,0.09,0.49}{{#1}}}
    \newcommand{\ControlFlowTok}[1]{\textcolor[rgb]{0.00,0.44,0.13}{\textbf{{#1}}}}
    \newcommand{\OperatorTok}[1]{\textcolor[rgb]{0.40,0.40,0.40}{{#1}}}
    \newcommand{\BuiltInTok}[1]{{#1}}
    \newcommand{\ExtensionTok}[1]{{#1}}
    \newcommand{\PreprocessorTok}[1]{\textcolor[rgb]{0.74,0.48,0.00}{{#1}}}
    \newcommand{\AttributeTok}[1]{\textcolor[rgb]{0.49,0.56,0.16}{{#1}}}
    \newcommand{\InformationTok}[1]{\textcolor[rgb]{0.38,0.63,0.69}{\textbf{\textit{{#1}}}}}
    \newcommand{\WarningTok}[1]{\textcolor[rgb]{0.38,0.63,0.69}{\textbf{\textit{{#1}}}}}
    
    
    % Define a nice break command that doesn't care if a line doesn't already
    % exist.
    \def\br{\hspace*{\fill} \\* }
    % Math Jax compatibility definitions
    \def\gt{>}
    \def\lt{<}
    \let\Oldtex\TeX
    \let\Oldlatex\LaTeX
    \renewcommand{\TeX}{\textrm{\Oldtex}}
    \renewcommand{\LaTeX}{\textrm{\Oldlatex}}
    % Document parameters
    % Document title
    \title{Advanced Statistical Methods for Modeling and Finance: Solutions for CTQs}
\author{Gbètoho Ezéchiel ADEDE}


% \date{\today}
    
    
    
    
    
% Pygments definitions
\makeatletter
\def\PY@reset{\let\PY@it=\relax \let\PY@bf=\relax%
    \let\PY@ul=\relax \let\PY@tc=\relax%
    \let\PY@bc=\relax \let\PY@ff=\relax}
\def\PY@tok#1{\csname PY@tok@#1\endcsname}
\def\PY@toks#1+{\ifx\relax#1\empty\else%
    \PY@tok{#1}\expandafter\PY@toks\fi}
\def\PY@do#1{\PY@bc{\PY@tc{\PY@ul{%
    \PY@it{\PY@bf{\PY@ff{#1}}}}}}}
\def\PY#1#2{\PY@reset\PY@toks#1+\relax+\PY@do{#2}}

\@namedef{PY@tok@w}{\def\PY@tc##1{\textcolor[rgb]{0.73,0.73,0.73}{##1}}}
\@namedef{PY@tok@c}{\let\PY@it=\textit\def\PY@tc##1{\textcolor[rgb]{0.24,0.48,0.48}{##1}}}
\@namedef{PY@tok@cp}{\def\PY@tc##1{\textcolor[rgb]{0.61,0.40,0.00}{##1}}}
\@namedef{PY@tok@k}{\let\PY@bf=\textbf\def\PY@tc##1{\textcolor[rgb]{0.00,0.50,0.00}{##1}}}
\@namedef{PY@tok@kp}{\def\PY@tc##1{\textcolor[rgb]{0.00,0.50,0.00}{##1}}}
\@namedef{PY@tok@kt}{\def\PY@tc##1{\textcolor[rgb]{0.69,0.00,0.25}{##1}}}
\@namedef{PY@tok@o}{\def\PY@tc##1{\textcolor[rgb]{0.40,0.40,0.40}{##1}}}
\@namedef{PY@tok@ow}{\let\PY@bf=\textbf\def\PY@tc##1{\textcolor[rgb]{0.67,0.13,1.00}{##1}}}
\@namedef{PY@tok@nb}{\def\PY@tc##1{\textcolor[rgb]{0.00,0.50,0.00}{##1}}}
\@namedef{PY@tok@nf}{\def\PY@tc##1{\textcolor[rgb]{0.00,0.00,1.00}{##1}}}
\@namedef{PY@tok@nc}{\let\PY@bf=\textbf\def\PY@tc##1{\textcolor[rgb]{0.00,0.00,1.00}{##1}}}
\@namedef{PY@tok@nn}{\let\PY@bf=\textbf\def\PY@tc##1{\textcolor[rgb]{0.00,0.00,1.00}{##1}}}
\@namedef{PY@tok@ne}{\let\PY@bf=\textbf\def\PY@tc##1{\textcolor[rgb]{0.80,0.25,0.22}{##1}}}
\@namedef{PY@tok@nv}{\def\PY@tc##1{\textcolor[rgb]{0.10,0.09,0.49}{##1}}}
\@namedef{PY@tok@no}{\def\PY@tc##1{\textcolor[rgb]{0.53,0.00,0.00}{##1}}}
\@namedef{PY@tok@nl}{\def\PY@tc##1{\textcolor[rgb]{0.46,0.46,0.00}{##1}}}
\@namedef{PY@tok@ni}{\let\PY@bf=\textbf\def\PY@tc##1{\textcolor[rgb]{0.44,0.44,0.44}{##1}}}
\@namedef{PY@tok@na}{\def\PY@tc##1{\textcolor[rgb]{0.41,0.47,0.13}{##1}}}
\@namedef{PY@tok@nt}{\let\PY@bf=\textbf\def\PY@tc##1{\textcolor[rgb]{0.00,0.50,0.00}{##1}}}
\@namedef{PY@tok@nd}{\def\PY@tc##1{\textcolor[rgb]{0.67,0.13,1.00}{##1}}}
\@namedef{PY@tok@s}{\def\PY@tc##1{\textcolor[rgb]{0.73,0.13,0.13}{##1}}}
\@namedef{PY@tok@sd}{\let\PY@it=\textit\def\PY@tc##1{\textcolor[rgb]{0.73,0.13,0.13}{##1}}}
\@namedef{PY@tok@si}{\let\PY@bf=\textbf\def\PY@tc##1{\textcolor[rgb]{0.64,0.35,0.47}{##1}}}
\@namedef{PY@tok@se}{\let\PY@bf=\textbf\def\PY@tc##1{\textcolor[rgb]{0.67,0.36,0.12}{##1}}}
\@namedef{PY@tok@sr}{\def\PY@tc##1{\textcolor[rgb]{0.64,0.35,0.47}{##1}}}
\@namedef{PY@tok@ss}{\def\PY@tc##1{\textcolor[rgb]{0.10,0.09,0.49}{##1}}}
\@namedef{PY@tok@sx}{\def\PY@tc##1{\textcolor[rgb]{0.00,0.50,0.00}{##1}}}
\@namedef{PY@tok@m}{\def\PY@tc##1{\textcolor[rgb]{0.40,0.40,0.40}{##1}}}
\@namedef{PY@tok@gh}{\let\PY@bf=\textbf\def\PY@tc##1{\textcolor[rgb]{0.00,0.00,0.50}{##1}}}
\@namedef{PY@tok@gu}{\let\PY@bf=\textbf\def\PY@tc##1{\textcolor[rgb]{0.50,0.00,0.50}{##1}}}
\@namedef{PY@tok@gd}{\def\PY@tc##1{\textcolor[rgb]{0.63,0.00,0.00}{##1}}}
\@namedef{PY@tok@gi}{\def\PY@tc##1{\textcolor[rgb]{0.00,0.52,0.00}{##1}}}
\@namedef{PY@tok@gr}{\def\PY@tc##1{\textcolor[rgb]{0.89,0.00,0.00}{##1}}}
\@namedef{PY@tok@ge}{\let\PY@it=\textit}
\@namedef{PY@tok@gs}{\let\PY@bf=\textbf}
\@namedef{PY@tok@gp}{\let\PY@bf=\textbf\def\PY@tc##1{\textcolor[rgb]{0.00,0.00,0.50}{##1}}}
\@namedef{PY@tok@go}{\def\PY@tc##1{\textcolor[rgb]{0.44,0.44,0.44}{##1}}}
\@namedef{PY@tok@gt}{\def\PY@tc##1{\textcolor[rgb]{0.00,0.27,0.87}{##1}}}
\@namedef{PY@tok@err}{\def\PY@bc##1{{\setlength{\fboxsep}{\string -\fboxrule}\fcolorbox[rgb]{1.00,0.00,0.00}{1,1,1}{\strut ##1}}}}
\@namedef{PY@tok@kc}{\let\PY@bf=\textbf\def\PY@tc##1{\textcolor[rgb]{0.00,0.50,0.00}{##1}}}
\@namedef{PY@tok@kd}{\let\PY@bf=\textbf\def\PY@tc##1{\textcolor[rgb]{0.00,0.50,0.00}{##1}}}
\@namedef{PY@tok@kn}{\let\PY@bf=\textbf\def\PY@tc##1{\textcolor[rgb]{0.00,0.50,0.00}{##1}}}
\@namedef{PY@tok@kr}{\let\PY@bf=\textbf\def\PY@tc##1{\textcolor[rgb]{0.00,0.50,0.00}{##1}}}
\@namedef{PY@tok@bp}{\def\PY@tc##1{\textcolor[rgb]{0.00,0.50,0.00}{##1}}}
\@namedef{PY@tok@fm}{\def\PY@tc##1{\textcolor[rgb]{0.00,0.00,1.00}{##1}}}
\@namedef{PY@tok@vc}{\def\PY@tc##1{\textcolor[rgb]{0.10,0.09,0.49}{##1}}}
\@namedef{PY@tok@vg}{\def\PY@tc##1{\textcolor[rgb]{0.10,0.09,0.49}{##1}}}
\@namedef{PY@tok@vi}{\def\PY@tc##1{\textcolor[rgb]{0.10,0.09,0.49}{##1}}}
\@namedef{PY@tok@vm}{\def\PY@tc##1{\textcolor[rgb]{0.10,0.09,0.49}{##1}}}
\@namedef{PY@tok@sa}{\def\PY@tc##1{\textcolor[rgb]{0.73,0.13,0.13}{##1}}}
\@namedef{PY@tok@sb}{\def\PY@tc##1{\textcolor[rgb]{0.73,0.13,0.13}{##1}}}
\@namedef{PY@tok@sc}{\def\PY@tc##1{\textcolor[rgb]{0.73,0.13,0.13}{##1}}}
\@namedef{PY@tok@dl}{\def\PY@tc##1{\textcolor[rgb]{0.73,0.13,0.13}{##1}}}
\@namedef{PY@tok@s2}{\def\PY@tc##1{\textcolor[rgb]{0.73,0.13,0.13}{##1}}}
\@namedef{PY@tok@sh}{\def\PY@tc##1{\textcolor[rgb]{0.73,0.13,0.13}{##1}}}
\@namedef{PY@tok@s1}{\def\PY@tc##1{\textcolor[rgb]{0.73,0.13,0.13}{##1}}}
\@namedef{PY@tok@mb}{\def\PY@tc##1{\textcolor[rgb]{0.40,0.40,0.40}{##1}}}
\@namedef{PY@tok@mf}{\def\PY@tc##1{\textcolor[rgb]{0.40,0.40,0.40}{##1}}}
\@namedef{PY@tok@mh}{\def\PY@tc##1{\textcolor[rgb]{0.40,0.40,0.40}{##1}}}
\@namedef{PY@tok@mi}{\def\PY@tc##1{\textcolor[rgb]{0.40,0.40,0.40}{##1}}}
\@namedef{PY@tok@il}{\def\PY@tc##1{\textcolor[rgb]{0.40,0.40,0.40}{##1}}}
\@namedef{PY@tok@mo}{\def\PY@tc##1{\textcolor[rgb]{0.40,0.40,0.40}{##1}}}
\@namedef{PY@tok@ch}{\let\PY@it=\textit\def\PY@tc##1{\textcolor[rgb]{0.24,0.48,0.48}{##1}}}
\@namedef{PY@tok@cm}{\let\PY@it=\textit\def\PY@tc##1{\textcolor[rgb]{0.24,0.48,0.48}{##1}}}
\@namedef{PY@tok@cpf}{\let\PY@it=\textit\def\PY@tc##1{\textcolor[rgb]{0.24,0.48,0.48}{##1}}}
\@namedef{PY@tok@c1}{\let\PY@it=\textit\def\PY@tc##1{\textcolor[rgb]{0.24,0.48,0.48}{##1}}}
\@namedef{PY@tok@cs}{\let\PY@it=\textit\def\PY@tc##1{\textcolor[rgb]{0.24,0.48,0.48}{##1}}}

\def\PYZbs{\char`\\}
\def\PYZus{\char`\_}
\def\PYZob{\char`\{}
\def\PYZcb{\char`\}}
\def\PYZca{\char`\^}
\def\PYZam{\char`\&}
\def\PYZlt{\char`\<}
\def\PYZgt{\char`\>}
\def\PYZsh{\char`\#}
\def\PYZpc{\char`\%}
\def\PYZdl{\char`\$}
\def\PYZhy{\char`\-}
\def\PYZsq{\char`\'}
\def\PYZdq{\char`\"}
\def\PYZti{\char`\~}
% for compatibility with earlier versions
\def\PYZat{@}
\def\PYZlb{[}
\def\PYZrb{]}
\makeatother


    % For linebreaks inside Verbatim environment from package fancyvrb. 
    \makeatletter
        \newbox\Wrappedcontinuationbox 
        \newbox\Wrappedvisiblespacebox 
        \newcommand*\Wrappedvisiblespace {\textcolor{red}{\textvisiblespace}} 
        \newcommand*\Wrappedcontinuationsymbol {\textcolor{red}{\llap{\tiny$\m@th\hookrightarrow$}}} 
        \newcommand*\Wrappedcontinuationindent {3ex } 
        \newcommand*\Wrappedafterbreak {\kern\Wrappedcontinuationindent\copy\Wrappedcontinuationbox} 
        % Take advantage of the already applied Pygments mark-up to insert 
        % potential linebreaks for TeX processing. 
        %        {, <, #, %, $, ' and ": go to next line. 
        %        _, }, ^, &, >, - and ~: stay at end of broken line. 
        % Use of \textquotesingle for straight quote. 
        \newcommand*\Wrappedbreaksatspecials {% 
            \def\PYGZus{\discretionary{\char`\_}{\Wrappedafterbreak}{\char`\_}}% 
            \def\PYGZob{\discretionary{}{\Wrappedafterbreak\char`\{}{\char`\{}}% 
            \def\PYGZcb{\discretionary{\char`\}}{\Wrappedafterbreak}{\char`\}}}% 
            \def\PYGZca{\discretionary{\char`\^}{\Wrappedafterbreak}{\char`\^}}% 
            \def\PYGZam{\discretionary{\char`\&}{\Wrappedafterbreak}{\char`\&}}% 
            \def\PYGZlt{\discretionary{}{\Wrappedafterbreak\char`\<}{\char`\<}}% 
            \def\PYGZgt{\discretionary{\char`\>}{\Wrappedafterbreak}{\char`\>}}% 
            \def\PYGZsh{\discretionary{}{\Wrappedafterbreak\char`\#}{\char`\#}}% 
            \def\PYGZpc{\discretionary{}{\Wrappedafterbreak\char`\%}{\char`\%}}% 
            \def\PYGZdl{\discretionary{}{\Wrappedafterbreak\char`\$}{\char`\$}}% 
            \def\PYGZhy{\discretionary{\char`\-}{\Wrappedafterbreak}{\char`\-}}% 
            \def\PYGZsq{\discretionary{}{\Wrappedafterbreak\textquotesingle}{\textquotesingle}}% 
            \def\PYGZdq{\discretionary{}{\Wrappedafterbreak\char`\"}{\char`\"}}% 
            \def\PYGZti{\discretionary{\char`\~}{\Wrappedafterbreak}{\char`\~}}% 
        } 
        % Some characters . , ; ? ! / are not pygmentized. 
        % This macro makes them "active" and they will insert potential linebreaks 
        \newcommand*\Wrappedbreaksatpunct {% 
            \lccode`\~`\.\lowercase{\def~}{\discretionary{\hbox{\char`\.}}{\Wrappedafterbreak}{\hbox{\char`\.}}}% 
            \lccode`\~`\,\lowercase{\def~}{\discretionary{\hbox{\char`\,}}{\Wrappedafterbreak}{\hbox{\char`\,}}}% 
            \lccode`\~`\;\lowercase{\def~}{\discretionary{\hbox{\char`\;}}{\Wrappedafterbreak}{\hbox{\char`\;}}}% 
            \lccode`\~`\:\lowercase{\def~}{\discretionary{\hbox{\char`\:}}{\Wrappedafterbreak}{\hbox{\char`\:}}}% 
            \lccode`\~`\?\lowercase{\def~}{\discretionary{\hbox{\char`\?}}{\Wrappedafterbreak}{\hbox{\char`\?}}}% 
            \lccode`\~`\!\lowercase{\def~}{\discretionary{\hbox{\char`\!}}{\Wrappedafterbreak}{\hbox{\char`\!}}}% 
            \lccode`\~`\/\lowercase{\def~}{\discretionary{\hbox{\char`\/}}{\Wrappedafterbreak}{\hbox{\char`\/}}}% 
            \catcode`\.\active
            \catcode`\,\active 
            \catcode`\;\active
            \catcode`\:\active
            \catcode`\?\active
            \catcode`\!\active
            \catcode`\/\active 
            \lccode`\~`\~ 	
        }
    \makeatother

    \let\OriginalVerbatim=\Verbatim
    \makeatletter
    \renewcommand{\Verbatim}[1][1]{%
        %\parskip\z@skip
        \sbox\Wrappedcontinuationbox {\Wrappedcontinuationsymbol}%
        \sbox\Wrappedvisiblespacebox {\FV@SetupFont\Wrappedvisiblespace}%
        \def\FancyVerbFormatLine ##1{\hsize\linewidth
            \vtop{\raggedright\hyphenpenalty\z@\exhyphenpenalty\z@
                \doublehyphendemerits\z@\finalhyphendemerits\z@
                \strut ##1\strut}%
        }%
        % If the linebreak is at a space, the latter will be displayed as visible
        % space at end of first line, and a continuation symbol starts next line.
        % Stretch/shrink are however usually zero for typewriter font.
        \def\FV@Space {%
            \nobreak\hskip\z@ plus\fontdimen3\font minus\fontdimen4\font
            \discretionary{\copy\Wrappedvisiblespacebox}{\Wrappedafterbreak}
            {\kern\fontdimen2\font}%
        }%
        
        % Allow breaks at special characters using \PYG... macros.
        \Wrappedbreaksatspecials
        % Breaks at punctuation characters . , ; ? ! and / need catcode=\active 	
        \OriginalVerbatim[#1,codes*=\Wrappedbreaksatpunct]%
    }
    \makeatother

    % Exact colors from NB
    \definecolor{incolor}{HTML}{303F9F}
    \definecolor{outcolor}{HTML}{D84315}
    \definecolor{cellborder}{HTML}{CFCFCF}
    \definecolor{cellbackground}{HTML}{F7F7F7}
    
    % prompt
    \makeatletter
    \newcommand{\boxspacing}{\kern\kvtcb@left@rule\kern\kvtcb@boxsep}
    \makeatother
    \newcommand{\prompt}[4]{
        {\ttfamily\llap{{\color{#2}[#3]:\hspace{3pt}#4}}\vspace{-\baselineskip}}
    }
    

    
    % Prevent overflowing lines due to hard-to-break entities
    \sloppy 
    % Setup hyperref package
    \hypersetup{
      breaklinks=true,  % so long urls are correctly broken across lines
      colorlinks=true,
      urlcolor=urlcolor,
      linkcolor=linkcolor,
      citecolor=citecolor,
      }
    % Slightly bigger margins than the latex defaults
    
    \geometry{verbose,tmargin=1in,bmargin=1in,lmargin=1in,rmargin=1in}

\usepackage{titleps}
\usepackage{fancyhdr}
\usepackage{graphicx}
\pagestyle{myheadings}
\pagestyle{fancy}
\fancyhf{}
\setlength{\headheight}{30pt}
\renewcommand{\headrulewidth}{4pt}
\renewcommand{\footrulewidth}{2pt}
\fancyhead[L]{\includegraphics[width=1cm]{example-image-a}}
\fancyhead[C]{}
\fancyhead[R]{\rightmark}
\fancyfoot[L]{CTQs}
\fancyfoot[C]{Gbètoho Ezéchiel ADEDE}
\fancyfoot[R]{\thepage}


%*****************************
\begin{document}
    
    \maketitle
    \newpage
    
    \section{CTQ 1}
For a roll of a pair of fair 6-sided dice, the sample space is defined as $\Omega = \{(i,j) , 1\leq i,j\leq 6\}$ and we have equiprobability. Then, the probability of any event equals the number of favorable cases divided by the number of possible cases, which is $card(\Omega) = 6 \times 6 = 36$

For each expression of \(X\), let's find the the probabilities $\mathbb {P}(X \lt 10)$, $\mathbb {P}(X \gt 40)$ and $\mathbb {P}$('X is a perfect square').

    
    \begin{tcolorbox}[breakable, size=fbox, boxrule=1pt, pad at break*=1mm,colback=cellbackground, colframe=cellborder]
\prompt{In}{incolor}{3}{\boxspacing}
\begin{Verbatim}[commandchars=\\\{\}]
\PY{k+kn}{import} \PY{n+nn}{numpy} \PY{k}{as} \PY{n+nn}{np}
\PY{k+kn}{import} \PY{n+nn}{math}
\end{Verbatim}
\end{tcolorbox}

    \begin{tcolorbox}[breakable, size=fbox, boxrule=1pt, pad at break*=1mm,colback=cellbackground, colframe=cellborder]
\prompt{In}{incolor}{30}{\boxspacing}
\begin{Verbatim}[commandchars=\\\{\}]
\PY{k}{def} \PY{n+nf}{solver}\PY{p}{(}\PY{n}{X}\PY{p}{)}\PY{p}{:}
    \PY{n}{a}\PY{o}{=}\PY{l+m+mi}{0}
    \PY{n+nb}{print}\PY{p}{(}\PY{p}{)}
    \PY{k}{for} \PY{n}{i} \PY{o+ow}{in} \PY{n+nb}{range}\PY{p}{(}\PY{l+m+mi}{6}\PY{p}{)}\PY{p}{:}
        \PY{k}{for} \PY{n}{j} \PY{o+ow}{in} \PY{n+nb}{range}\PY{p}{(}\PY{l+m+mi}{6}\PY{p}{)}\PY{p}{:}
            \PY{k}{if} \PY{n}{X}\PY{p}{[}\PY{n}{i}\PY{p}{,}\PY{n}{j}\PY{p}{]}\PY{o}{\PYZlt{}}\PY{l+m+mi}{10}\PY{p}{:}
                \PY{n}{a}\PY{o}{+}\PY{o}{=}\PY{l+m+mi}{1}
                \PY{n+nb}{print}\PY{p}{(}\PY{p}{(}\PY{n}{i}\PY{o}{+}\PY{l+m+mi}{1}\PY{p}{,}\PY{n}{j}\PY{o}{+}\PY{l+m+mi}{1}\PY{p}{)}\PY{p}{,}\PY{l+s+s1}{\PYZsq{}}\PY{l+s+s1}{ ; }\PY{l+s+s1}{\PYZsq{}}\PY{p}{,}\PY{n}{end}\PY{o}{=}\PY{l+s+s1}{\PYZsq{}}\PY{l+s+s1}{\PYZsq{}}\PY{p}{)}
    \PY{n+nb}{print}\PY{p}{(}\PY{l+s+s1}{\PYZsq{}}\PY{l+s+se}{\PYZbs{}n}\PY{l+s+s1}{Number of favorable cases:}\PY{l+s+s1}{\PYZsq{}}\PY{p}{,}\PY{n}{a}\PY{p}{)}
    \PY{n+nb}{print}\PY{p}{(}\PY{l+s+s1}{\PYZsq{}}\PY{l+s+s1}{************************************}\PY{l+s+s1}{\PYZsq{}}\PY{p}{)} 
    \PY{n}{a}\PY{o}{=}\PY{l+m+mi}{0}
    \PY{k}{for} \PY{n}{i} \PY{o+ow}{in} \PY{n+nb}{range}\PY{p}{(}\PY{l+m+mi}{6}\PY{p}{)}\PY{p}{:}
        \PY{k}{for} \PY{n}{j} \PY{o+ow}{in} \PY{n+nb}{range}\PY{p}{(}\PY{l+m+mi}{6}\PY{p}{)}\PY{p}{:}
            \PY{k}{if} \PY{n}{X}\PY{p}{[}\PY{n}{i}\PY{p}{,}\PY{n}{j}\PY{p}{]}\PY{o}{\PYZgt{}}\PY{l+m+mi}{40}\PY{p}{:}
                \PY{n}{a}\PY{o}{+}\PY{o}{=}\PY{l+m+mi}{1}
                \PY{n+nb}{print}\PY{p}{(}\PY{p}{(}\PY{n}{i}\PY{o}{+}\PY{l+m+mi}{1}\PY{p}{,}\PY{n}{j}\PY{o}{+}\PY{l+m+mi}{1}\PY{p}{)}\PY{p}{,}\PY{l+s+s1}{\PYZsq{}}\PY{l+s+s1}{ ; }\PY{l+s+s1}{\PYZsq{}}\PY{p}{,}\PY{n}{end}\PY{o}{=}\PY{l+s+s1}{\PYZsq{}}\PY{l+s+s1}{\PYZsq{}}\PY{p}{)}
    \PY{n+nb}{print}\PY{p}{(}\PY{l+s+s1}{\PYZsq{}}\PY{l+s+se}{\PYZbs{}n}\PY{l+s+s1}{Number of favorable cases:}\PY{l+s+s1}{\PYZsq{}}\PY{p}{,}\PY{n}{a}\PY{p}{)}           
    \PY{n+nb}{print}\PY{p}{(}\PY{l+s+s1}{\PYZsq{}}\PY{l+s+s1}{************************************}\PY{l+s+s1}{\PYZsq{}}\PY{p}{)} 
    \PY{n}{a}\PY{o}{=}\PY{l+m+mi}{0}
    \PY{k}{for} \PY{n}{i} \PY{o+ow}{in} \PY{n+nb}{range}\PY{p}{(}\PY{l+m+mi}{6}\PY{p}{)}\PY{p}{:}
        \PY{k}{for} \PY{n}{j} \PY{o+ow}{in} \PY{n+nb}{range}\PY{p}{(}\PY{l+m+mi}{6}\PY{p}{)}\PY{p}{:}
            \PY{k}{if} \PY{n}{math}\PY{o}{.}\PY{n}{isqrt}\PY{p}{(}\PY{n}{X}\PY{p}{[}\PY{n}{i}\PY{p}{,}\PY{n}{j}\PY{p}{]}\PY{p}{)}\PY{o}{*}\PY{o}{*}\PY{l+m+mi}{2}\PY{o}{==}\PY{n}{X}\PY{p}{[}\PY{n}{i}\PY{p}{,}\PY{n}{j}\PY{p}{]}\PY{p}{:}
                \PY{n}{a}\PY{o}{+}\PY{o}{=}\PY{l+m+mi}{1}
                \PY{n+nb}{print}\PY{p}{(}\PY{p}{(}\PY{n}{i}\PY{o}{+}\PY{l+m+mi}{1}\PY{p}{,}\PY{n}{j}\PY{o}{+}\PY{l+m+mi}{1}\PY{p}{)}\PY{p}{,}\PY{l+s+s1}{\PYZsq{}}\PY{l+s+s1}{ ; }\PY{l+s+s1}{\PYZsq{}}\PY{p}{,}\PY{n}{end}\PY{o}{=}\PY{l+s+s1}{\PYZsq{}}\PY{l+s+s1}{\PYZsq{}}\PY{p}{)}
    \PY{n+nb}{print}\PY{p}{(}\PY{l+s+s1}{\PYZsq{}}\PY{l+s+se}{\PYZbs{}n}\PY{l+s+s1}{Number of favorable cases:}\PY{l+s+s1}{\PYZsq{}}\PY{p}{,}\PY{n}{a}\PY{p}{)}   
                
\PY{n}{X}\PY{o}{=}\PY{n}{np}\PY{o}{.}\PY{n}{zeros}\PY{p}{(}\PY{p}{(}\PY{l+m+mi}{6}\PY{p}{,}\PY{l+m+mi}{6}\PY{p}{)}\PY{p}{,}\PY{n}{dtype}\PY{o}{=}\PY{n+nb}{int}\PY{p}{)}                
\end{Verbatim}
\end{tcolorbox}

    \hypertarget{xijiuxb2juxb2}{%
\subsection{X(i,j)=i²+j²}\label{xijiuxb2juxb2}}

    \begin{tcolorbox}[breakable, size=fbox, boxrule=1pt, pad at break*=1mm,colback=cellbackground, colframe=cellborder]
\prompt{In}{incolor}{31}{\boxspacing}
\begin{Verbatim}[commandchars=\\\{\}]
\PY{k}{for} \PY{n}{i} \PY{o+ow}{in} \PY{n+nb}{range}\PY{p}{(}\PY{l+m+mi}{6}\PY{p}{)}\PY{p}{:}
    \PY{k}{for} \PY{n}{j} \PY{o+ow}{in} \PY{n+nb}{range}\PY{p}{(}\PY{l+m+mi}{6}\PY{p}{)}\PY{p}{:}
        \PY{n}{X}\PY{p}{[}\PY{n}{i}\PY{p}{,}\PY{n}{j}\PY{p}{]}\PY{o}{=}\PY{p}{(}\PY{n}{i}\PY{o}{+}\PY{l+m+mi}{1}\PY{p}{)}\PY{o}{*}\PY{o}{*}\PY{l+m+mi}{2}\PY{o}{+}\PY{p}{(}\PY{n}{j}\PY{o}{+}\PY{l+m+mi}{1}\PY{p}{)}\PY{o}{*}\PY{o}{*}\PY{l+m+mi}{2}
\PY{n+nb}{print}\PY{p}{(}\PY{n}{X}\PY{p}{)}
\PY{n}{solver}\PY{p}{(}\PY{n}{X}\PY{p}{)}
\end{Verbatim}
\end{tcolorbox}

    \begin{Verbatim}[commandchars=\\\{\}]
[[ 2  5 10 17 26 37]
 [ 5  8 13 20 29 40]
 [10 13 18 25 34 45]
 [17 20 25 32 41 52]
 [26 29 34 41 50 61]
 [37 40 45 52 61 72]]

(1, 1)  ; (1, 2)  ; (2, 1)  ; (2, 2)  ;
Number of favorable cases: 4
************************************
(3, 6)  ; (4, 5)  ; (4, 6)  ; (5, 4)  ; (5, 5)  ; (5, 6)  ; (6, 3)  ; (6, 4)  ;
(6, 5)  ; (6, 6)  ;
Number of favorable cases: 10
************************************
(3, 4)  ; (4, 3)  ;
Number of favorable cases: 2
    \end{Verbatim}
\subsubsection{Probability 1: \(\mathbb {P}(X < 10)\)}

%\begin{equation}
%X < 10 \iff \quad (i, j) \in \{(1,1) ; (1,2) ; (2,1) ; (2,2)\}
%\end{equation}

\(\mathbb {P}(X < 10)\)=$\frac{4}{36}$=$\frac{1}{9}$


\subsubsection{Probability 2: \(\mathbb {P}(X > 40)\)}

%\begin{equation}
%X > 40 \iff \quad (i, j) \in \{(3, 6) ; (4, 5) ; (4, 6) ; (5, 4) ; (5, 5) ;(5, 6) ; (6, 3) ; (6, 4) ; (6, 5) ; (6, 6)\}
%¨\end{equation}

\(\mathbb {P}(X > 40)\)=$\frac{10}{36}$=$\frac{5}{18}$

\subsubsection{Probability 3: The probability that X is a perfect square}

%X is a perfect square $\iff \quad (i, j) \in \%{(3, 4) ; (4, 3)\}$

$\mathbb {P}$("X is a perfect square")=$\frac{2}{36}$=$\frac{1}{18}$


    \hypertarget{xijiuxb2j}{%
\subsection{X(i,j)=i²+j}\label{xijiuxb2j}}

    \begin{tcolorbox}[breakable, size=fbox, boxrule=1pt, pad at break*=1mm,colback=cellbackground, colframe=cellborder]
\prompt{In}{incolor}{32}{\boxspacing}
\begin{Verbatim}[commandchars=\\\{\}]
\PY{k}{for} \PY{n}{i} \PY{o+ow}{in} \PY{n+nb}{range}\PY{p}{(}\PY{l+m+mi}{6}\PY{p}{)}\PY{p}{:}
    \PY{k}{for} \PY{n}{j} \PY{o+ow}{in} \PY{n+nb}{range}\PY{p}{(}\PY{l+m+mi}{6}\PY{p}{)}\PY{p}{:}
        \PY{n}{X}\PY{p}{[}\PY{n}{i}\PY{p}{,}\PY{n}{j}\PY{p}{]}\PY{o}{=}\PY{p}{(}\PY{n}{i}\PY{o}{+}\PY{l+m+mi}{1}\PY{p}{)}\PY{o}{*}\PY{o}{*}\PY{l+m+mi}{2}\PY{o}{+}\PY{p}{(}\PY{n}{j}\PY{o}{+}\PY{l+m+mi}{1}\PY{p}{)}
\PY{n+nb}{print}\PY{p}{(}\PY{n}{X}\PY{p}{)}
\PY{n}{solver}\PY{p}{(}\PY{n}{X}\PY{p}{)}
\end{Verbatim}
\end{tcolorbox}

    \begin{Verbatim}[commandchars=\\\{\}]
[[ 2  3  4  5  6  7]
 [ 5  6  7  8  9 10]
 [10 11 12 13 14 15]
 [17 18 19 20 21 22]
 [26 27 28 29 30 31]
 [37 38 39 40 41 42]]

(1, 1)  ; (1, 2)  ; (1, 3)  ; (1, 4)  ; (1, 5)  ; (1, 6)  ; (2, 1)  ; (2, 2)  ;
(2, 3)  ; (2, 4)  ; (2, 5)  ;
Number of favorable cases: 11
************************************
(6, 5)  ; (6, 6)  ;
Number of favorable cases: 2
************************************
(1, 3)  ; (2, 5)  ;
Number of favorable cases: 2
    \end{Verbatim}
    
    \subsubsection{Probability 1: $\mathbb {P}$\((X < 10)\)}

%\begin{equation}
%X < 10 \iff \quad (i, j) \in \{(1,1) ; (1,2) ; (2,1) ; (2,2)\}
%\end{equation}

$\mathbb {P}$\((X < 10)\)=$\frac{11}{36}$


\subsubsection{Probability 2: $\mathbb {P}$\((X > 40)\)}

%\begin{equation}
%X > 40 \iff \quad (i, j) \in \{(3, 6) ; (4, 5) ; (4, 6) ; (5, 4) ; (5, 5) ;(5, 6) ; (6, 3) ; (6, 4) ; (6, 5) ; (6, 6)\}
%¨\end{equation}

$\mathbb {P}$\((X > 40)\)=$\frac{2}{36}$=$\frac{1}{9}$

\subsubsection{Probability 3: The probability that X is a perfect square}

%X is a perfect square $\iff \quad (i, j) \in \%{(3, 4) ; (4, 3)\}$

$\mathbb {P}$("X is a perfect square")=$\frac{2}{36}$=$\frac{1}{18}$

    \hypertarget{xijijuxb2}{%
\subsection{X(i,j)=i+j²}\label{xijijuxb2}}

    \begin{tcolorbox}[breakable, size=fbox, boxrule=1pt, pad at break*=1mm,colback=cellbackground, colframe=cellborder]
\prompt{In}{incolor}{33}{\boxspacing}
\begin{Verbatim}[commandchars=\\\{\}]
\PY{k}{for} \PY{n}{i} \PY{o+ow}{in} \PY{n+nb}{range}\PY{p}{(}\PY{l+m+mi}{6}\PY{p}{)}\PY{p}{:}
    \PY{k}{for} \PY{n}{j} \PY{o+ow}{in} \PY{n+nb}{range}\PY{p}{(}\PY{l+m+mi}{6}\PY{p}{)}\PY{p}{:}
        \PY{n}{X}\PY{p}{[}\PY{n}{i}\PY{p}{,}\PY{n}{j}\PY{p}{]}\PY{o}{=}\PY{p}{(}\PY{n}{i}\PY{o}{+}\PY{l+m+mi}{1}\PY{p}{)}\PY{o}{+}\PY{p}{(}\PY{n}{j}\PY{o}{+}\PY{l+m+mi}{1}\PY{p}{)}\PY{o}{*}\PY{o}{*}\PY{l+m+mi}{2}
\PY{n+nb}{print}\PY{p}{(}\PY{n}{X}\PY{p}{)}
\PY{n}{solver}\PY{p}{(}\PY{n}{X}\PY{p}{)}
\end{Verbatim}
\end{tcolorbox}

    \begin{Verbatim}[commandchars=\\\{\}]
[[ 2  5 10 17 26 37]
 [ 3  6 11 18 27 38]
 [ 4  7 12 19 28 39]
 [ 5  8 13 20 29 40]
 [ 6  9 14 21 30 41]
 [ 7 10 15 22 31 42]]

(1, 1)  ; (1, 2)  ; (2, 1)  ; (2, 2)  ; (3, 1)  ; (3, 2)  ; (4, 1)  ; (4, 2)  ;
(5, 1)  ; (5, 2)  ; (6, 1)  ;
Number of favorable cases: 11
************************************
(5, 6)  ; (6, 6)  ;
Number of favorable cases: 2
************************************
(3, 1)  ; (5, 2)  ;
Number of favorable cases: 2
    \end{Verbatim}
    
       \subsubsection{Probability 1: $\mathbb {P}$\((X < 10)\)}

%\begin{equation}
%X < 10 \iff \quad (i, j) \in \{(1,1) ; (1,2) ; (2,1) ; (2,2)\}
%\end{equation}

$\mathbb {P}$\((X < 10)\)=$\frac{11}{36}$


\subsubsection{Probability 2: $\mathbb {P}$\((X > 40)\)}

%\begin{equation}
%X > 40 \iff \quad (i, j) \in \{(3, 6) ; (4, 5) ; (4, 6) ; (5, 4) ; (5, 5) ;(5, 6) ; (6, 3) ; (6, 4) ; (6, 5) ; (6, 6)\}
%¨\end{equation}

$\mathbb {P}$\((X > 40)\)=$\frac{2}{36}$=$\frac{1}{9}$

\subsubsection{Probability 3: The probability that X is a perfect square}

%X is a perfect square $\iff \quad (i, j) \in \%{(3, 4) ; (4, 3)\}$

$\mathbb {P}$("X is a perfect square")=$\frac{2}{36}$=$\frac{1}{18}$


    \hypertarget{xijijuxb2}{%
\subsection{X(i,j)=(i+j)²}\label{xijijuxb2}}

    \begin{tcolorbox}[breakable, size=fbox, boxrule=1pt, pad at break*=1mm,colback=cellbackground, colframe=cellborder]
\prompt{In}{incolor}{34}{\boxspacing}
\begin{Verbatim}[commandchars=\\\{\}]
\PY{k}{for} \PY{n}{i} \PY{o+ow}{in} \PY{n+nb}{range}\PY{p}{(}\PY{l+m+mi}{6}\PY{p}{)}\PY{p}{:}
    \PY{k}{for} \PY{n}{j} \PY{o+ow}{in} \PY{n+nb}{range}\PY{p}{(}\PY{l+m+mi}{6}\PY{p}{)}\PY{p}{:}
        \PY{n}{X}\PY{p}{[}\PY{n}{i}\PY{p}{,}\PY{n}{j}\PY{p}{]}\PY{o}{=}\PY{p}{(}\PY{n}{i}\PY{o}{+}\PY{n}{j}\PY{o}{+}\PY{l+m+mi}{2}\PY{p}{)}\PY{o}{*}\PY{o}{*}\PY{l+m+mi}{2}
\PY{n+nb}{print}\PY{p}{(}\PY{n}{X}\PY{p}{)}
\PY{n}{solver}\PY{p}{(}\PY{n}{X}\PY{p}{)}
\end{Verbatim}
\end{tcolorbox}

    \begin{Verbatim}[commandchars=\\\{\}]
[[  4   9  16  25  36  49]
 [  9  16  25  36  49  64]
 [ 16  25  36  49  64  81]
 [ 25  36  49  64  81 100]
 [ 36  49  64  81 100 121]
 [ 49  64  81 100 121 144]]

(1, 1)  ; (1, 2)  ; (2, 1)  ;
Number of favorable cases: 3
************************************
(1, 6)  ; (2, 5)  ; (2, 6)  ; (3, 4)  ; (3, 5)  ; (3, 6)  ; (4, 3)  ; (4, 4)  ;
(4, 5)  ; (4, 6)  ; (5, 2)  ; (5, 3)  ; (5, 4)  ; (5, 5)  ; (5, 6)  ; (6, 1)  ;
(6, 2)  ; (6, 3)  ; (6, 4)  ; (6, 5)  ; (6, 6)  ;
Number of favorable cases: 21
************************************
(1, 1)  ; (1, 2)  ; (1, 3)  ; (1, 4)  ; (1, 5)  ; (1, 6)  ; (2, 1)  ; (2, 2)  ;
(2, 3)  ; (2, 4)  ; (2, 5)  ; (2, 6)  ; (3, 1)  ; (3, 2)  ; (3, 3)  ; (3, 4)  ;
(3, 5)  ; (3, 6)  ; (4, 1)  ; (4, 2)  ; (4, 3)  ; (4, 4)  ; (4, 5)  ; (4, 6)  ;
(5, 1)  ; (5, 2)  ; (5, 3)  ; (5, 4)  ; (5, 5)  ; (5, 6)  ; (6, 1)  ; (6, 2)  ;
(6, 3)  ; (6, 4)  ; (6, 5)  ; (6, 6)  ;
Number of favorable cases: 36
    \end{Verbatim}

\subsubsection{Probability 1: $\mathbb {P}$\((X < 10)\)}

%\begin{equation}
%X < 10 \iff \quad (i, j) \in \{(1,1) ; (1,2) ; (2,1) ; (2,2)\}
%\end{equation}

$\mathbb {P}$\((X < 10)\)=$\frac{3}{36}$=$\frac{1}{12}$


\subsubsection{Probability 2: $\mathbb {P}$\((X > 40)\)}

%\begin{equation}
%X > 40 \iff \quad (i, j) \in \{(3, 6) ; (4, 5) ; (4, 6) ; (5, 4) ; (5, 5) ;(5, 6) ; (6, 3) ; (6, 4) ; (6, 5) ; (6, 6)\}
%¨\end{equation}

$\mathbb {P}$\((X > 40)\)=$\frac{21}{36}$=$\frac{7}{12}$

\subsubsection{Probability 3: The probability that X is a perfect square}

%X is a perfect square $\iff \quad (i, j) \in \%{(3, 4) ; (4, 3)\}$

$\mathbb {P}$("X is a perfect square")=$\frac{36}{36}$=$1$

\newpage

\section{CTQ 2}
\subsection{Jacobian matrix: Definition}

Let $m$ and $n$ be two non-zero natural numbers and
\begin{align*}
f\colon \mathbb{R}^m & \longrightarrow\mathbb{R}^n\\
x&\longmapsto f(x)=(f_i(x))_{1 \leq i \leq n}
\end{align*}

a vectorial function which admits all its first-order partial derivates.

The Jacobian matrix of \( f \) is the matrix with \( n \) rows and \( m \) columns defined by
\[ J_f(x) = \left(\frac{\partial f_i}{\partial x_j}(x) \right)_{\substack{1 \leq i \leq n \\ 1 \leq j \leq m}} \forall x=(x_k)_{1 \leq k \leq m} \]    

\subsection{Use of Jacobian matrix for random variables transformation(Multivariate Transformation Method)}

We can use the Jacobian matrix to determine the joint PDF of a multivariate distribution $Y$ that is a reversible function of another multivariate distribution $X$ (i.e. $X$ is a transformation of $Y$) knowing the joint \textbf{PDF}  of $X$.

 \subsubsection{Theorem}
Let $X_1, X_2, \ldots, X_n$ $n$ continous random variables, $\mathbf{X} = (X_1, X_2, \ldots, X_n)$ and $\mathbf{Z} = g(\mathbf{X}) = (g_1(\mathbf{X}), g_2(\mathbf{X}), \ldots, g_n(\mathbf{X}))= (Z_1, Z_2, \ldots, Z_n)$ with $g$ a reversible function with continous partial derivates. 

Let $\mathbf{W} = g^{-1}(\mathbf{Z}) = (h_1(\mathbf{Z}), h_2(\mathbf{Z}), \ldots, h_n(\mathbf{Z}))= (W_1, W_2, \ldots, W_n)$.

The joint \textbf{PDF} of $\mathbf{Z}$ is given by:

\[
f_{\mathbf{Z}}(\mathbf{z}) = f_{\mathbf{X}}(\mathbf{h}(\mathbf{z})) \cdot |\mathbf{J_h(z)}|
\]

where:
\begin{itemize}
  \item $f_{\mathbf{X}}$ is the joint \textbf{PDF} of $\mathbf{X}$.
  \item $\mathbf{h} = (h_1, h_2, \ldots, h_n)$ is the inverse transformation function.
  \item $|\mathbf{J_h}|$ is the determinant of the Jacobian matrix $\mathbf{J_h}$ of the inverse transformation(inverse of the transformation g)
\end{itemize}

\subsubsection{Illustration example}

Let $X=(X_1,X_2)$ be a couple of idd continous random variables following the standard normal distribution.

Let another couple of continous random variables $Y=(Y_1,Y_2)$ such that  
\begin{align*}
\begin{cases}
            Y_1=2X_1-X_2  \\
            Y_2=-X_1+X_2\\ 
        \end{cases}   \\
\end{align*}

Let $g$ the transformation function. We have

$g(X_1,X_2)=(2X_1-X_2,-X_1+X_2)$

Let  us find $X=(X_1,X_2)$ such that $g(X)=Y=(Y_1,Y_2)$

\begin{align*}
g(X)=Y=(Y_1,Y_2) 
& \iff \begin{cases}
            Y_1=2X_1-X_2  \\
            Y_2=-X_1+X_2\\ 
        \end{cases}   \\        
& \iff \begin{cases}
            -X_1+X_2=Y_2  \\
            X_1=Y_1+Y_2\\ 
        \end{cases}   \\        
& \iff \begin{cases}
            X_1=Y_1+Y_2 \\
            X_2=X_1+Y_2\\ 
        \end{cases}   \\
& \iff \begin{cases}
            X_1=Y_1+Y_2 \\
            X_2=Y_1+Y_2+Y_2\\ 
        \end{cases}   \\
& \iff \begin{cases}
            X_1=Y_1+Y_2 \\
            X_2=Y_1+2Y_2\\ 
        \end{cases}   \\
\end{align*}
Then $g$ is inversible and 

$g^{-1}(X_1,X_2)=h(X_1,X_2)=(X_1+X_2,X_1+2X_2 )=(h_1(X_1,X_2),h_1(X_1,X_2))$

$\frac{\partial h_1}{\partial x_1}(x_1,x_2)=1 $
 
$\frac{\partial h_1}{\partial x_2}(x_1,x_2)=1 $ 
 
$\frac{\partial h_2}{\partial x_1}(x_1,x_2)=1 $ 
 
$\frac{\partial h_2}{\partial x_2}(x_1,x_2)=2 $

Then $|J_h(x1,x2)|=\begin{vmatrix}
1 & 1 \\
1 & 2 \\
\end{vmatrix}=2-1=1$

Let $(x_1,x_2) \in \mathbf{R}^2  $

The joint \textbf{PDF} of $\mathbf{Y}$ is given by:

\[
f_Y((x_1,x_2)) = f_X(h(x_1,x_2)) \cdot |J_h(x_1,x_2)|
\]
\begin{align*}
f_X(x_1,x_2)
&=f_{X_1}(x_1)\times f_{X_2}(x_2) \text{ bacause  } X_1\perp X_2\\
&=\frac{1}{\sqrt{2\pi}} e^{-\frac{x_1^2}{2}}\times \frac{1}{\sqrt{2\pi}} e^{-\frac{x_2^2}{2}}\\
&=\frac{1}{2\pi} e^{-\frac{x_1^2}{2} -\frac{x_2^2}{2}}\\
&=\frac{1}{2\pi} e^{-\frac{1}{2}(x_1^2+x_2^2)}\\
\end{align*}
$h(x_1,x_2)=(x_1+x_2,x_1+2x_2 )$
\begin{align*}
f_X(h(x_1,x_2))
&=f_X(x_1+x_2,x_1+2x_2 )\\
&=\frac{1}{2\pi} e^{-\frac{1}{2}[(x_1+x_2)^2+(x_1+2x_2)^2]}\\
&=\frac{1}{2\pi} e^{-\frac{1}{2}[x_1^2+ 2x_1 x_2+x_2^2+x_1^2+ 4x_1 x_2+4x_2^2]}\\
&=\frac{1}{2\pi} e^{-\frac{1}{2}[x_1^2+ x_1^2+ 2x_1 x_2+4 x_1 x_2 +x_2^2+ 4x_2^2]}\\
&=\frac{1}{2\pi} e^{-\frac{1}{2}[2x_1^2+ 6x_1 x_2+ 5x_2^2]}\\
\end{align*}
And then we have:
\begin{align*}
f_Y(x_1,x_2)
&=\frac{1}{2\pi} e^{-\frac{1}{2}[2x_1^2+ 6x_1 x_2+ 5x_2^2]} \times 1\\
&=\frac{1}{2\pi} e^{-\frac{1}{2}(2x_1^2+ 6x_1 x_2+ 5x_2^2)} \\
\end{align*}

\newpage


\section{CTQ 3}
\subsection{Chebyshev's inequality}
\subsubsection{Statment}

Let X be a random variable.

If $E(X)$ and $V(X)$ exist, then $\forall a \gt 0 $, $\mathbb {P}(\left| X-E(X) \right|\ge a )\le \frac{V(X)}{a^2}$
\subsubsection{Proof}

Let $Y=(X-E(X))^2$ and $ a \gt 0$

$\left| X-E(X) \right| \ge a \iff  (X-E(X))^2\ge a^2 \iff Y \ge a^2$ 

Then $\mathbb {P}(\left| X-E(X) \right| \ge a)=\mathbb {P}(Y \ge a^2)$



Y is non-negative  and $E(Y)=E((X-E(X))^2)=Var(X)$ exist .

Consequently, by using Markov's inequality with $c=a^2$ and $u(X)=Y$, we have

$\mathbb {P}(Y\ge a^2) \le \frac{E(Y)}{a^2}$
 
Since $\mathbb {P}(\left| X-E(X) \right| \ge a)=P(Y \ge a^2)$  and 
$E(Y)=E((X-E(X))^2)$ we have

$\mathbb {P}(\left| X-E(X) \right| \ge a )\le \frac{V(X)}{a^2}$

\subsection{}

Let $h \gt 0$  and \(X\) be a random variable with MGF $M_X$ defined on $\left]-h, h\right[$



\begin{itemize}
\item[•] Let $ 0 \lt t \lt h$\\
Let us prove that $\mathbb {P}(X \geq a) \leq  exp(-at)M_X(t)$
\begin{align*}
X \ge a &\iff tX \ge ta \text{ (because t \gt 0)} \\
        &\iff e^{tX} \ge e^{ta} \\
\end{align*}
Then we have
\begin{align*}
\mathbb {P}(X \geq a) =\mathbb {P}((e^{tX} \ge e^{ta}) \\
\end{align*}
Put $U(X)=e^{tX} \text{ and }c=e^{ta} $.

As $U(X) \gt 0 \text{ and } c\gt 0$, we can apply Markov's inequality and then we obtain
\begin{align*}
\mathbb {P}(X \geq a) &=\mathbb {P}((e^{tX} \ge e^{ta}) \\
            &\leq \frac{E(e^{tX})}{e^{ta}}  \\
            &= e^{-ta}M_X(t)  \\
\end{align*}
Hence $P(X \geq a) \leq e^{-ta}M_X(t)$


\item[•] Let's  $ -h\lt t \lt 0 $\\
Let us prove that $\mathbb {P}(X \leq a) \leq  exp(-at)M_X(t)$
\begin{align*}
X \le a &\iff tX \ge ta \text{ (because t \lt 0)} \\
        &\iff e^{tX} \ge e^{ta} \\
\end{align*}
Then we have
\begin{align*}
\mathbb {P}(X \leq a) =\mathbb {P}(e^{tX} \ge e^{ta}) \\
\end{align*}
Put $U(X)=e^{tX} \text{ and }c=e^{ta} $.

As $U(X) \gt 0 \text{ and } c\gt 0$, we can apply Markov's inequality and then we obtain
\begin{align*}
\mathbb {P}(X \leq a) &=\mathbb {P}(e^{tX} \ge e^{ta}) \\
            &\leq \frac{E(e^{tX})}{e^{ta}}  \\
            &= e^{-ta}M_X(t)  \\
\end{align*}
Hence $\mathbb {P}(X \leq a) \leq e^{-ta}M_X(t)$
\end{itemize}

\newpage

\section{CTQ 4}

Let X be a random variable having 

\begin{itemize}
\item[•]$\mu$ as expectation
\item[•]$\sigma$ as standard deviation
\item[•]$M_X$ as \textbf{MGF}, such that 

$M_X(t)=(\frac{2}{3}+\frac{1}{3}e^t)^9 \textbf{ }\forall t\in \mathbb{R}$
\end{itemize}

\subsection{ Let us show that 
$\mathbb {P}(\mu -2\sigma \lt  X \lt \mu+2\sigma)=\Sigma_{x=1}^{5}\binom{9}{x}(\frac{1}{3})^x(\frac{2}{3})^{9-x}$}

Let $F_X$ be the \textbf{CDF} of X. Then we have
\begin{align*}
\mathbb {P}((\mu -2\sigma \lt  X \lt \mu+2\sigma) &= F_X(\mu+2\sigma)-F_X(\mu -2\sigma) \\
\end{align*}
Let $t\in \mathbb{R}$
We have
\begin{align*}
M_X(t) &=(\frac{2}{3}+\frac{1}{3}e^t)^9 \\
       &=(1-\frac{1}{3}+\frac{1}{3}e^t)^9 \\
       &=(1-p+pe^t)^n \text{ with } p=\frac{1}{3} \text{ and } n=9\\
       &=M_Y(t)\text{ where } Y \sim \mathcal B(n=9,p=\frac{1}{3})
\end{align*} 
Then $X$ and $Y$ has the same MGF and consequently, they are the same and then we have:
\begin{align*}
\mathbb {P}(\mu -2\sigma \lt  X \lt \mu+2\sigma) &= F_Y(\mu+2\sigma)-F_Y(\mu -2\sigma) \text{ with } F_Y \text{ the \textbf{CDF} of Y } \\
\end{align*}
\begin{align*}
\mu &= np\\
    &= 9\times \frac{1}{3}\\
    &=3
\end{align*}
\begin{align*}
\sigma &= \sqrt{npq} \\
       &= \sqrt{np(1-p)} \\ 
       &= \sqrt{9\times \frac{1}{3}\times \frac{2}{3}} \\ 
       &= \sqrt{2} \\          
\end{align*}
$Y(\omega)=\{0, 1, \ldots, 9\}$ and
 $F_Y(y)=\mathbb {P}((Y\lt y)=\Sigma_{i=1}^{j}{ \mathbb {P}((Y=y_i)}   \textbf{ } \forall  y\in \left]y_j ; y_{j+1}\right] $
\begin{align*}
\mu -2\sigma &= 3-2\sqrt{2} \\
       &\approx 0.1 \in \left]0 ; 1\right]=\left]y_1 ; y_2\right]         
\end{align*}
Then 
\begin{align*}
F_Y(\mu -2\sigma) &= \mathbb {P}((Y=y_1) \\           
\end{align*}
\begin{align*}
\mu +2\sigma &= 3+2\sqrt{2} \\
       &\approx 5.82 \in \left]5 ; 6\right]=\left]y_6 ; y_7\right] \\          
\end{align*}

Then 
\begin{align*}
F_Y(\mu +2\sigma) &= \Sigma_{i=1}^{6}{ \mathbb {P}((Y=y_i)}   \\          
\end{align*}
Consequently we have 
\begin{align*}
\mathbb {P}(\mu -2\sigma \lt  X \lt \mu+2\sigma) &=(\Sigma_{i=1}^{6}{ \mathbb {P}((Y=y_i)}) -\mathbb {P}((Y=y_i)  \\
&=\Sigma_{i=2}^{6}{ \mathbb {P}(Y=y_i)}  \\
\end{align*}
\begin{align*}
Y(\omega)&=\{0, 1, \ldots, 9\}\\ 
         &=\{y_1, y_2, \ldots, y_{10}\} 
\end{align*}
So $y_i=i-1 \forall i \in \{1, 2, \ldots, 6\} $ and then $y_i=i-1 \forall i \in \{2, 3, \ldots, 6\} $

Consequently, we have:
\begin{align*}
\mathbb {P}(\mu -2\sigma \lt  X \lt \mu+2\sigma)
&=\Sigma_{i=2}^{6}{\mathbb {P}((Y=i-1)}  \\
\end{align*}
Let $k=i-1$

$i=2 \Rightarrow k=1$ and $i=6 \Rightarrow k=5$

Hence, as $\mathbb {P}(X=k)=\binom{9}{k}(\frac{1}{3})^k(\frac{2}{3})^{9-k}$, we get  
\begin{align*}
\mathbb {P}(\mu -2\sigma \lt  X \lt \mu+2\sigma)
&=\Sigma_{k=1}^{5}{\mathbb {P}(Y=k)}  \\
&=\Sigma_{k=1}^{5}{ \binom{9}{k}(\frac{1}{3})^k(\frac{2}{3})^{9-k}}  \\
\end{align*}

\subsection{Let us compute that probability using R programming language}


 \begin{tcolorbox}[breakable, size=fbox, boxrule=1pt, pad at break*=1mm,colback=cellbackground, colframe=cellborder]
\prompt{In}{incolor}{19}{\boxspacing}
\begin{Verbatim}[commandchars=\\\{\}]
\PY{c+c1}{\PYZsh{}R code in jupyter notebook}
\PY{n}{n}\PY{o}{=} \PY{l+m}{9}
\PY{n}{p}\PY{o}{=} \PY{l+m}{1}\PY{o}{/}\PY{l+m}{3}

\PY{n}{mu}\PY{o}{=} \PY{n}{n} \PY{o}{*} \PY{n}{p}  \PY{c+c1}{\PYZsh{} Mean}
\PY{n}{sigma} \PY{o}{=}\PY{n+nf}{sqrt}\PY{p}{(}\PY{n}{n} \PY{o}{*} \PY{n}{p} \PY{o}{*} \PY{p}{(}\PY{l+m}{1} \PY{o}{\PYZhy{}} \PY{n}{p}\PY{p}{)}\PY{p}{)}  \PY{c+c1}{\PYZsh{} Standard deviation}

\PY{c+c1}{\PYZsh{}pbinom is the CDF}
\PY{n}{proba}\PY{o}{=} \PY{n+nf}{pbinom}\PY{p}{(}\PY{n}{mu} \PY{o}{+} \PY{l+m}{2} \PY{o}{*} \PY{n}{sigma}\PY{p}{,} \PY{n}{size} \PY{o}{=} \PY{n}{n}\PY{p}{,} \PY{n}{prob} \PY{o}{=} \PY{n}{p}\PY{p}{)} \PY{o}{\PYZhy{}} \PY{n+nf}{pbinom}\PY{p}{(}\PY{n}{mu} \PY{o}{\PYZhy{}} \PY{l+m}{2} \PY{o}{*} \PY{n}{sigma}\PY{p}{,} \PY{n}{size} \PY{o}{=} \PY{n}{n}\PY{p}{,} \PY{n}{prob} \PY{o}{=} \PY{n}{p}\PY{p}{)}
\PY{n}{proba}
\end{Verbatim}
\end{tcolorbox}

    0.931565310166133

\newpage

 \section{CTQ 5}
    
   
   Let X be a random variable that has a Poisson ditribution with $\lambda$ as parameter
   
   We have:
   
    $\mathbb {P}(X=k)=\frac{\lambda^{k} e^{-\lambda}}{k! } \text{ for k in } \mathbb{N}$
   
   
   Let $M_X$ be the \textbf{MGF} of $X$
  
  \subsection{Let us find $M_X$} 

  Let $t \in \mathbb{R}$
  \begin{align*}
M_X(t)&=E(e^tX)\\
	&=\Sigma_{k\in \mathbb{N}  }e^{tk}\mathbb{P}(X=k)  \\
	&=\Sigma_{k\in \mathbb{N}  }e^{tk}\frac{\lambda^{k} e^{-\lambda}}{k!}  \\
	&=\Sigma_{k\in \mathbb{N}  }\frac{e^{tk}\lambda^{k} e^{-\lambda}}{k!}  \\
	 &=\Sigma_{k\in \mathbb{N}  }\frac{(e^{t})^k\lambda^{k} e^{-\lambda}}{k!}  \\
	 &=\Sigma_{k\in \mathbb{N}  }\frac{(\lambda e^{t})^k e^{-\lambda}}{k!}  \\
\end{align*}
$\forall x \in \mathbb{R}\text{, } e^x= \Sigma_{n\in \mathbb{N}  }\frac{x^{n}}{n!}$

Then we have:
\begin{align*}
M_X(t)&=e^{-\lambda}e^{\lambda e^t}\\
	&=e^{-\lambda+\lambda e^t}\\
	&=e^{\lambda e^t-\lambda}\\
	&=e^{\lambda( e^t-1)}
\end{align*}
Hence $M_X(t)=e^{\lambda( e^t-1)}$
   
\subsection{Put $Y=\frac{X-\lambda}{\sqrt{\lambda}}$ \\Let us show that when $\lambda$ is large, $M_Y(t)$ goes to $e^{ \frac{t^2}{2} }  \text{ }\forall t \in \mathbb{R\\
}$}

Let $t \in \mathbb{R}$
\begin{align*}
M_Y(t)&=M_{\frac{X-\lambda}{\sqrt{\lambda}}}(t)\\
	&=E[exp(\frac{t}{\sqrt{\lambda}}(X-\lambda)]\\
		&=E[exp(\frac{tX}{\sqrt{\lambda}}-\frac{\lambda}{\sqrt{\lambda}}t)]\\
	&=E[exp(\frac{tX}{\sqrt{\lambda}}-\sqrt{\lambda}t)]\\
	&=E[exp(\frac{tX}{\sqrt{\lambda}})\times exp(-\sqrt{\lambda}t)]\\
	&= exp(-\sqrt{\lambda}t)E[exp(\frac{tX}{\sqrt{\lambda}})] \text{ because } -\sqrt{\lambda}t \text{ doesn't depend on } X 
\end{align*}
$M_X(y)=E[exp(yX)]\textbf{ } \forall y \in \mathbb{R} $

For $y=\frac{t}{\sqrt{\lambda}}$ we have

$M_X(\frac{t}{\sqrt{\lambda}})=E[exp(\frac{tX}{\sqrt{\lambda}})]$

Then, since $M_X(t)=e^{\lambda( e^t-1)}$ we have
\begin{align*}
M_Y(t)&=exp(-\sqrt{\lambda}t)E[exp(\frac{tX}{\sqrt{\lambda}})]\\	
	&=exp(-\sqrt{\lambda}t)  M_X(\frac{t}{\sqrt{\lambda}})\\ 
	&=exp(-\sqrt{\lambda}t)  exp[\lambda(exp(\frac{t}{\sqrt{\lambda}})-1)]\\ 
\end{align*}
\begin{align*}
exp(\frac{t}{\sqrt{\lambda}})&=\sum_{n=0}^{\infty} \frac{x^n}{n!}\\	
	&=1+\sum_{n=1}^{\infty} \frac{x^n}{n!}\\ 
	&=exp(-\sqrt{\lambda}t)  exp[\lambda(exp(\frac{t}{\sqrt{\lambda}})-1)]\\ 
\end{align*}
\begin{align*}
exp(\frac{t}{\sqrt{\lambda}})-1&=\sum_{n=0}^{\infty} \frac{x^n}{n!}\\	
	&=1+\sum_{n=1}^{\infty} \frac{({\frac{t}{\sqrt{\lambda}})}^{n}}{n!}\\ 
	&=1+[\sum_{n=1}^{\infty} \frac{t^n}{ {(\sqrt{\lambda})}^n n!}]-1\\
	&=\sum_{n=1}^{\infty} \frac{t^n}{ {(\sqrt{\lambda})}^n n!}\\ 
\end{align*}
\begin{align*}
\lambda(exp(\frac{t}{\sqrt{\lambda}})-1)&=\lambda\sum_{n=1}^{\infty} \frac{t^n}{ {(\sqrt{\lambda})}^n n!}\\
&=\sum_{n=1}^{\infty} \frac{\lambda t^n}{ {(\sqrt{\lambda})}^n n!}\\	
&=\sum_{n=1}^{\infty} \frac{\lambda t^n}{ {({\lambda}^ \frac{1}{2})}^n n!}\\
&=\sum_{n=1}^{\infty} \frac{\lambda t^n}{ {({\lambda} )}^\frac{n}{2} n!}\\
&=\sum_{n=1}^{\infty} \frac{{({\lambda} )}^{1-{\frac{n}{2}}} t^n}{  n!}\\
&=\sum_{n=1}^{\infty} \frac{{({\lambda} )}^{\frac{2-n}{2}} t^n}{  n!}\\
&=\frac{{\lambda}^{\frac{1}{2}}t}{1!} + \frac{{\lambda}^{0}t^2}{2!} +\sum_{n=3}^{\infty} \frac{{({\lambda} )}^{\frac{2-n}{2}} t^n}{  n!}\\
&={\lambda}^{\frac{1}{2}}t +\frac{t^2}{2}+ \sum_{n=3}^{\infty} \frac{{({\lambda} )}^{\frac{2-n}{2}} t^n}{  n!}\\
&=\sqrt{\lambda}t +\frac{t^2}{2}+ \sum_{n=3}^{\infty} \frac{{({\lambda} )}^{\frac{2-n}{2}} t^n}{  n!}\\
\end{align*}
\begin{align*}
exp[\lambda(exp(\frac{t}{\sqrt{\lambda}})-1)]&=exp[\sqrt{\lambda}t +\frac{t^2}{2}+ \sum_{n=3}^{\infty} \frac{{({\lambda} )}^{\frac{2-n}{2}} t^n}{  n!}]\\
&=exp(\sqrt{\lambda}t) \times exp(\frac{t^2}{2}) \times exp[ \sum_{n=3}^{\infty} \frac{{({\lambda} )}^{\frac{2-n}{2}} t^n}{  n!}]\\
\end{align*}
\begin{align*}
M_Y(t)&=exp(-\sqrt{\lambda}t)  exp[\lambda(exp(\frac{t}{\sqrt{\lambda}})-1)]\\  
&=exp(-\sqrt{\lambda}t) \times   exp(\sqrt{\lambda}t) \times exp(\frac{t^2}{2}) \times exp[ \sum_{n=3}^{\infty} \frac{{({\lambda} )}^{\frac{2-n}{2}} t^n}{  n!}]\\  
&=exp(\frac{t^2}{2}) \times exp[ \sum_{n=3}^{\infty} \frac{{({\lambda} )}^{\frac{2-n}{2}} t^n}{  n!}]\\ 
\end{align*}
Let $ n \in \mathbb{N}$
\begin{align*}
n \ge 3 &\iff n \gt 2  \\  
&\Rightarrow -n \lt -2  \\
&\Rightarrow 2-n \lt 0  \\
&\Rightarrow \frac{2-n}{2}  \lt 0 \\
\end{align*}

So $\lim_{{\lambda \to +\infty}} (\lambda)^{ \frac{2-n}{2}}=0\textbf{ } \forall n \ge 3 $

Then 

$\lim_{{\lambda \to +\infty}} \sum_{n=3}^{\infty} \frac{{({\lambda} )}^{\frac{2-n}{2}} t^n}{  n!}=0 $

And then

$\lim_{{\lambda \to +\infty}} exp[\sum_{n=3}^{\infty} \frac{{({\lambda} )}^{\frac{2-n}{2}} t^n}{  n!}]=1 $

And finaly

$\lim_{{\lambda \to +\infty}} M_Y(t)=exp(t^2)$

    % Add a bibliography block to the postdoc
  
  Hence when $\lambda$ is too large and  $Y=\frac{X-\lambda}{\sqrt{\lambda}}$, $M_Y(t)$ goes to $e^{ \frac{t^2}{2} }  \text{ }\forall t \in \mathbb{R}$ 
  
  
 \subsection{Conclusion} 
 
 Let $Z$ be a randow variable following the standard normal distribution. Then the \textbf{MGF} of $Z$ is defined by  $M_Z(t)=e^{ \frac{t^2}{2} }$ 
  
  And Then, when $\lambda$ is large, $Y=\frac{X-\lambda}{\sqrt{\lambda}}$ follows the standard normal distribution and consequently $ X \sim \mathcal N(\lambda,\sqrt{\lambda})$ 
  
  \subsection{Let us illustrate the convergence with a Python programm} 
  
      \begin{tcolorbox}[breakable, size=fbox, boxrule=1pt, pad at break*=1mm,colback=cellbackground, colframe=cellborder]
\prompt{In}{incolor}{16}{\boxspacing}
\begin{Verbatim}[commandchars=\\\{\}]
\PY{k+kn}{import} \PY{n+nn}{numpy} \PY{k}{as} \PY{n+nn}{np}
\PY{k+kn}{import} \PY{n+nn}{matplotlib}\PY{n+nn}{.}\PY{n+nn}{pyplot} \PY{k}{as} \PY{n+nn}{plt}
\PY{n}{lambda\PYZus{}params} \PY{o}{=} \PY{p}{[}\PY{l+m+mf}{0.5}\PY{p}{,}\PY{l+m+mi}{1}\PY{p}{,}\PY{l+m+mi}{5}\PY{p}{,}\PY{l+m+mi}{10}\PY{p}{,}\PY{l+m+mi}{25}\PY{p}{,}\PY{l+m+mi}{50}\PY{p}{,}\PY{l+m+mi}{100}\PY{p}{,}\PY{l+m+mi}{1000}\PY{p}{]}

\PY{c+c1}{\PYZsh{} Générer des valeurs possibles}
\PY{c+c1}{\PYZsh{}X = np.linspace(\PYZhy{}100, 100,num=100)}
\PY{n}{n}\PY{o}{=}\PY{l+m+mi}{10000}

\PY{c+c1}{\PYZsh{}help(np.random.normal)}
\PY{k}{def} \PY{n+nf}{func}\PY{p}{(}\PY{n}{lambda\PYZus{}params}\PY{p}{)}\PY{p}{:}
    \PY{n}{num}\PY{o}{=}\PY{l+m+mi}{1}
    \PY{n}{plt}\PY{o}{.}\PY{n}{figure}\PY{p}{(}\PY{n}{figsize}\PY{o}{=}\PY{p}{(}\PY{l+m+mi}{10}\PY{p}{,} \PY{l+m+mi}{8}\PY{p}{)}\PY{p}{)}
    \PY{k}{for} \PY{n}{i} \PY{o+ow}{in} \PY{n+nb}{range}\PY{p}{(}\PY{n+nb}{len}\PY{p}{(}\PY{n}{lambda\PYZus{}params}\PY{p}{)}\PY{p}{)}\PY{p}{:}
        
        \PY{n}{lambda\PYZus{}param}\PY{o}{=}\PY{n}{lambda\PYZus{}params}\PY{p}{[}\PY{n}{i}\PY{p}{]}
       
        \PY{n}{pmf\PYZus{}poisson} \PY{o}{=} \PY{n}{np}\PY{o}{.}\PY{n}{random}\PY{o}{.}\PY{n}{poisson}\PY{p}{(}\PY{n}{lambda\PYZus{}param}\PY{p}{,}\PY{n}{size}\PY{o}{=}\PY{n}{n}\PY{p}{)}
        \PY{n}{pdf\PYZus{}norm}  \PY{o}{=} \PY{n}{np}\PY{o}{.}\PY{n}{random}\PY{o}{.}\PY{n}{normal}\PY{p}{(}\PY{n}{loc}\PY{o}{=}\PY{n}{lambda\PYZus{}param}\PY{p}{,}\PY{n}{size}\PY{o}{=}\PY{n}{n}\PY{p}{,}\PY{n}{scale}\PY{o}{=}\PY{n}{np}\PY{o}{.}\PY{n}{sqrt}\PY{p}{(}\PY{n}{lambda\PYZus{}param}\PY{p}{)}\PY{p}{)}

        \PY{n}{plt}\PY{o}{.}\PY{n}{subplot}\PY{p}{(}\PY{n+nb}{len}\PY{p}{(}\PY{n}{lambda\PYZus{}params}\PY{p}{)}\PY{p}{,}\PY{l+m+mi}{2}\PY{p}{,}\PY{n}{num}\PY{p}{)}
        \PY{n}{plt}\PY{o}{.}\PY{n}{hist}\PY{p}{(}\PY{n}{pmf\PYZus{}poisson}\PY{p}{,}\PY{l+m+mi}{25}\PY{p}{,}\PY{n}{color}\PY{o}{=}\PY{l+s+s1}{\PYZsq{}}\PY{l+s+s1}{m}\PY{l+s+s1}{\PYZsq{}}\PY{p}{,}\PY{n}{label}\PY{o}{=}\PY{l+s+s1}{\PYZsq{}}\PY{l+s+s1}{Poisson distribution}\PY{l+s+s1}{\PYZsq{}}\PY{p}{,}\PY{n}{density}\PY{o}{=}\PY{k+kc}{True}\PY{p}{)}
        \PY{n}{plt}\PY{o}{.}\PY{n}{title}\PY{p}{(}\PY{l+s+s1}{\PYZsq{}}\PY{l+s+s1}{lamb=}\PY{l+s+s1}{\PYZsq{}}\PY{o}{+}\PY{n+nb}{str}\PY{p}{(}\PY{n}{lambda\PYZus{}param}\PY{p}{)}\PY{p}{)}
        
        \PY{n}{plt}\PY{o}{.}\PY{n}{subplot}\PY{p}{(}\PY{n+nb}{len}\PY{p}{(}\PY{n}{lambda\PYZus{}params}\PY{p}{)}\PY{p}{,}\PY{l+m+mi}{2}\PY{p}{,}\PY{n}{num}\PY{o}{+}\PY{l+m+mi}{1}\PY{p}{)}
        \PY{n}{plt}\PY{o}{.}\PY{n}{hist}\PY{p}{(}\PY{n}{pdf\PYZus{}norm}\PY{p}{,}\PY{l+m+mi}{25}\PY{p}{,}\PY{n}{color}\PY{o}{=}\PY{l+s+s1}{\PYZsq{}}\PY{l+s+s1}{blue}\PY{l+s+s1}{\PYZsq{}}\PY{p}{,}\PY{n}{label}\PY{o}{=}\PY{l+s+s1}{\PYZsq{}}\PY{l+s+s1}{Normal distribution}\PY{l+s+s1}{\PYZsq{}}\PY{p}{,}\PY{n}{density}\PY{o}{=}\PY{k+kc}{True}\PY{p}{)}
        \PY{n}{plt}\PY{o}{.}\PY{n}{title}\PY{p}{(}\PY{l+s+s1}{\PYZsq{}}\PY{l+s+s1}{lam=}\PY{l+s+s1}{\PYZsq{}}\PY{o}{+}\PY{n+nb}{str}\PY{p}{(}\PY{n}{lambda\PYZus{}param}\PY{p}{)}\PY{p}{)}
        \PY{n}{num}\PY{o}{+}\PY{o}{=}\PY{l+m+mi}{2}
        \PY{c+c1}{\PYZsh{}plt.legend(loc=\PYZsq{}best\PYZsq{})}

        \PY{n}{plt}\PY{o}{.}\PY{n}{suptitle}\PY{p}{(}\PY{l+s+s1}{\PYZsq{}}\PY{l+s+s1}{Convergence of the Poisson distribution to the normal distribution}\PY{l+s+s1}{\PYZsq{}}\PY{p}{)}
    \PY{n}{plt}\PY{o}{.}\PY{n}{show}\PY{p}{(}\PY{p}{)}

\PY{n}{func}\PY{p}{(}\PY{n}{lambda\PYZus{}params}\PY{p}{)}
\end{Verbatim}
\end{tcolorbox}

    \begin{center}
    \adjustimage{max size={0.9\linewidth}{0.9\paperheight}}{convergence.png}
    \end{center}
    { \hspace*{\fill} \\}

\newpage
     
  
  \section{CTQ 6} 
  
  Let $ X \sim \mathcal U(0,1)$ with $f_X$ as \textbf{PDF} and $F_X$ as \textbf{CDF}
  
   Let $ Y=e^X$ with $f_Y$ as \textbf{PDF} and $F_Y$ as \textbf{CDF}
   
   Let us find $F_Y$
   
   Let  $t \in \mathbb{R}$
\begin{align*}
F_Y(t)
&=\mathbb {P}(Y \le t)\\  
&=\mathbb {P}(e^X \le t)\\ 
&=  \begin{cases}
            0 & \text{if } t \le 0 \\
            \mathbb {P}(X \le\ln t)& \text{else } 
        \end{cases}   \\
        &=  \begin{cases}
            0 & \text{if } t \le 0 \\
            F_X(\ln t) & \text{else } 
        \end{cases}   \\
\end{align*}
\begin{align*}
f_Y(t)
&=F_Y'(t)\\
&=  \begin{cases}
            0 & \text{if } t \le 0 \\
            \frac{d}{dt} [F_X(\ln t)]& \text{else } 
        \end{cases}   \\
        &=  \begin{cases}
            0 & \text{if } t \le 0 \\
            \frac{1}{t} \times F_X'(\ln t)& \text{else } 
        \end{cases}   \\
        &=  \begin{cases}
            0 & \text{if } t \le 0 \\
            \frac{1}{t} \times f_X(\ln t)& \text{else } 
        \end{cases}   \\  
\end{align*}
\begin{align*}
 \forall t\gt 0 \text{, } f_X(\ln t)
&=  \begin{cases}
            1 & \text{if } 0 \le \ln t \le 1 \\
            0& \text{else } 
        \end{cases}   \\
 &=  \begin{cases}
            1 & \text{if } 1 \le t \le e \\
            0& \text{if t  } \in ]0;1[\cup]e;+\infty[ 
        \end{cases}   \\
\end{align*}
And then 
\begin{align*}
f_X(t)
&=  \begin{cases}
		   0 & \text{if } t \le 0 \\ 
           \frac{1}{t} & \text{if } 1 \le t \le e \\
           0& \text{if t  } \in ]0;1[\cup]e;+\infty[
        \end{cases}   \\  
        &=  \begin{cases}
        \frac{1}{t} & \text{if } 1 \le t \le e \\
		   0 & \text{else } \\ 
        \end{cases}   \\            
\end{align*}

\newpage  
  \section{CTQ 7} 
  
  Let X be a randon variable having as \textbf{MGF} the function defined by $M_X(t)=e^{3t+8t^2} \text{ } \forall t \in \mathbb{R}$
 
 Let us find $\mathbb {P}((-1\lt X\lt 9)$
 \begin{align*}
M_X(t)&=e^{3t+8t^2}\\  
&=e^{\mu t+\sigma \frac{t^2}{2} } \text{ with } \mu=3 \text{ and }  \frac{\sigma^2}{2} =8 \text{ ie } \sigma^2=16 \text{ and then }  \sigma=4  \\   
\end{align*}
Then  $ X \sim \mathcal N(\mu,\sigma)$ and then $Z=\frac{X-\mu}{\sigma}$ follows the standard normal distribution and we have $X=\mu +\sigma Z$ 

\begin{align*}
\mathbb {P}(-1\lt X\lt 9)
&=\mathbb {P}(-1\lt \mu +\sigma Z \lt 9)\\  
&=\mathbb {P}(\frac{-1-\mu}{\sigma} \lt  Z \lt \frac{9-\mu}{\sigma})\\ 
&=\mathbb {P}(\frac{-1-3}{4} \lt  Z \lt \frac{9-3}{4})\\ 
&=\mathbb {P}(-1 \lt  Z \lt \frac{6}{4})\\ 
&=\mathbb {P}(-1 \lt  Z \lt \frac{3}{2})\\ 
&=\Phi(\frac{3}{2}) - \Phi(-1)\\ 
&=\Phi(\frac{3}{2}) - (1-\Phi(1))\\
&=\Phi(\frac{3}{2}) - 1+\Phi(1)\\ 
&=\Phi(1)+\Phi(\frac{3}{2}) - 1\\ 
\end{align*} 
 \begin{align*}
\Phi(1)
&=\Phi(1.0+0.00)\\
&=0.8413   
\end{align*} 

 \begin{align*}
\Phi(\frac{3}{2})
&=\Phi(1.5)\\
&=\Phi(1.5+0.00)\\
&=0.9332\\   
\end{align*}
 \begin{align*}
\mathbb {P}(-1\lt X\lt 9)
&=0.8413 +0.9332-1\\  
&=0.7745\\  
\end{align*} 

\newpage

\section{CTQ 8}
Let us choose  \textbf{Weibull distribution}.

Let X be a random varible. 

$X$ is said to follow the Weibulll distribution with parameters $k$ and $\lambda$ when it's PDF is defined by

\begin{align*}
f_X(t)  
        &=  \begin{cases}
        \frac{k}{\lambda}{(\frac{t}{\lambda})}^{k-1}exp({-\frac{t}{\lambda})}^{k} & \text{if } t \gt 0 \\
		   0 & \text{else } \\ 
        \end{cases}   \\            
\end{align*}

\begin{itemize}
\item[•]$k \gt 0$ is the \textbf{shape}  parameter
\item[•] $\lambda \gt 0$ is the \textbf{scale} of the distribution
\end{itemize}


\subsection{MGF of Weibull distribution}

Let $t \in \mathbb{R}$ and $M_X$ the mgf of X

\begin{align*}
M_X(t)  
        &=  E[e^{tX}]\\ 
        &= \int_{-\infty}^{+\infty} e^{tx}f(x)dx \\            
             &=0+ \int_{0}^{+\infty} e^{tx}(\frac{k}{\lambda}){(\frac{x}{\lambda})}^{k-1}e^{(-\frac{x}{\lambda})^{k}}dx \\
             &=\int_{0}^{+\infty} e^{tx}(\frac{k}{\lambda}){(\frac{x}{\lambda})}^{k-1}e^{(-\frac{x}{\lambda})^{k}}dx \\
\end{align*}
Put $u=\frac{x}{\lambda}$. Then $x=\lambda u$ and $dx=\lambda du$.

And then we have

\begin{align*}
M_X(t)  
        &=\int_{0}^{+\infty} e^{\lambda tu}(\frac{k}{\lambda})u^{k-1}e^{(-u)^{k}}(\lambda)du \\
         &=\int_{0}^{+\infty} e^{\lambda tu}(k)u^{k-1}e^{(-u)^{k}}du \\
\end{align*}

Put $x=u^k$. Then $dx=u^{k-1}du$  and $u=x^{\frac{1}{k}}$.

And then we have

\begin{align*}
M_X(t)  
        &=\int_{0}^{+\infty} e^{\lambda t(x)^{\frac{1}{k}}}e^{-x}dx \\
\end{align*}

$\forall x  \in [0;\infty[$, we have:


\begin{align*}
e^{\lambda t(x)^{\frac{1}{k}}}  
        &=\sum_{n=0}^{+\infty} \frac{[{\lambda t(x)^{\frac{1}{k}}}]^n}{n!} \\
         &=\sum_{n=0}^{+\infty} \frac{({\lambda t})^n}{n!} (x)^{\frac{n}{k}} \\
\end{align*}
So we have
\begin{align*}
M_X(t)  
        &=\int_{0}^{+\infty} [\sum_{n=0}^{+\infty} \frac{({\lambda t})^n}{n!} (x)^{\frac{n}{k}}] e^{-x}dx \\
        &=\int_{0}^{+\infty} (\sum_{n=0}^{+\infty} \frac{({\lambda t})^n}{n!} (x)^{\frac{n}{k}} e^{-x})dx \\
        &=\sum_{n=0}^{+\infty} (\int_{0}^{+\infty} \frac{({\lambda t})^n}{n!} (x)^{\frac{n}{k}} e^{-x})dx \\
        &=\sum_{n=0}^{+\infty} \frac{({\lambda t})^n}{n!} \int_{0}^{+\infty}  (x)^{\frac{n}{k}} e^{-x}dx \\
\end{align*}
The Gamma function is defined by $\Gamma(y)=\int_{0}^{+\infty} x^{y-1} e^{-x}dx \text{ } \forall y \in \mathbb{R}$

$y-1=\frac{n}{k} \iff y=1+\frac{n}{k}$

So  $\int_{0}^{+\infty}  (x)^{\frac{n}{k}} e^{-x}dx=\Gamma(1+\frac{n}{k})$ and then
\begin{align*}
M_X(t)  
        &=\sum_{n=0}^{+\infty} \frac{({\lambda t})^n}{n!} \Gamma(1+\frac{n}{k}) \\
        &=\sum_{n=0}^{+\infty} \frac{{\lambda }^n}{n!} \Gamma(1+\frac{n}{k}) t^n \\
\end{align*}


\subsection{Mean and variance of Weibull distribution}

\begin{itemize}
\item[•]Mean $E(X)$


$E(X)=M_X'(0)$
\begin{align*}
M_X(t)  
        &=\sum_{n=0}^{+\infty} \frac{{\lambda }^n}{n!} \Gamma(1+\frac{n}{k}) t^n \\
        &=\frac{{\lambda }^0}{0!} \Gamma(1+\frac{0}{k}) t^0+\sum_{n=1}^{+\infty} \frac{{\lambda }^n}{n!} \Gamma(1+\frac{n}{k}) t^n \\
         &= \Gamma(1) +\sum_{n=1}^{+\infty} \frac{{\lambda }^n}{n!} \Gamma(1+\frac{n}{k}) t^n \\
\end{align*}
\begin{align*}
M_X'(t)  
        &=0+\sum_{n=1}^{+\infty} \frac{{\lambda }^n}{n!} \Gamma(1+\frac{n}{k}) nt^{n-1} \\
         &=\sum_{n=1}^{+\infty} \frac{{\lambda }^n}{n(n-1)!} \Gamma(1+\frac{n}{k}) nt^{n-1} \\
         &=\sum_{n=1}^{+\infty} \frac{{\lambda }^n}{(n-1)!} \Gamma(1+\frac{n}{k})t^{n-1} \\
         &=\frac{{\lambda }^1}{(1-1)!} \Gamma(1+\frac{1}{k})t^{1-1}+\sum_{n=2}^{+\infty} \frac{{\lambda }^n}{(n-1)!} \Gamma(1+\frac{n}{k})t^{n-1} \\
         &=\lambda \Gamma(1+\frac{1}{k})+\sum_{n=2}^{+\infty} \frac{{\lambda }^n}{(n-1)!} \Gamma(1+\frac{n}{k})t^{n-1} \\
\end{align*}
$\forall n \geq 2 \textbf{, } n\gt 1 \textbf{ ie }  n-1 \gt 0$ and then we have
\begin{align*}
M_X'(0)  
         &=\lambda \Gamma(1+\frac{1}{k})+\sum_{n=2}^{+\infty} \frac{{\lambda }^n}{(n-1)!} \Gamma(1+\frac{n}{k})0^{n-1} \\
         &=\lambda \Gamma(1+\frac{1}{k})+0 \\
         &=\lambda \Gamma(1+\frac{1}{k}) \\
\end{align*}
Hence $E(X)=\lambda \Gamma(1+\frac{1}{k})$

\item[•]Variance $Var(X)$

$Var(X)=E(X^2)-[E(X)]^2$

$E(X^2)=M_X''(0)$
\end{itemize}

\begin{align*}
M_X'(t)  
         &=\lambda \Gamma(1+\frac{1}{k})+\sum_{n=2}^{+\infty} \frac{{\lambda }^n}{(n-1)!} \Gamma(1+\frac{n}{k})t^{n-1} \\
         &=\lambda \Gamma(1+\frac{1}{k})+\frac{{\lambda }^2}{(2-1)!} \Gamma(1+\frac{2}{k})t^{2-1}+\sum_{n=3}^{+\infty} \frac{{\lambda }^n}{(n-1)!} \Gamma(1+\frac{n}{k})t^{n-1} \\
         &=\lambda \Gamma(1+\frac{1}{k})+{\lambda }^2 \Gamma(1+\frac{2}{k})t+\sum_{n=3}^{+\infty} \frac{{\lambda }^n}{(n-1)!} \Gamma(1+\frac{n}{k})t^{n-1} \\
\end{align*}
\begin{align*}
M_X''(t)  
         &=0+{\lambda }^2 \Gamma(1+\frac{2}{k})+\sum_{n=3}^{+\infty} \frac{{\lambda }^n}{(n-1)!} \Gamma(1+\frac{n}{k})(n-1)t^{n-2} \\
         &={\lambda }^2 \Gamma(1+\frac{2}{k})+\sum_{n=3}^{+\infty} \frac{{\lambda }^n}{(n-1)(n-2)!} \Gamma(1+\frac{n}{k})(n-1)t^{n-2} \\
         &={\lambda }^2 \Gamma(1+\frac{2}{k})+\sum_{n=3}^{+\infty} \frac{{\lambda }^n}{(n-2)!} \Gamma(1+\frac{n}{k})t^{n-2} \\
\end{align*}
$\forall n \geq 3 \textbf{, } n\gt 3 \textbf{ ie }  n-3 \gt 0$ and then we have
\begin{align*}
M_X''(0)  
         &={\lambda }^2 \Gamma(1+\frac{2}{k})+\sum_{n=3}^{+\infty} \frac{{\lambda }^n}{(n-2)!} \Gamma(1+\frac{n}{k})0^{n-2} \\
         &={\lambda }^2 \Gamma(1+\frac{2}{k})+0 \\
         &={\lambda }^2 \Gamma(1+\frac{2}{k})
\end{align*}
Hence 

$E(X^2)={\lambda }^2 \Gamma(1+\frac{2}{k})$
\begin{align*}
Var(X)  
         &=E(X^2)-[E(X)]^2\\
         &={\lambda }^2 \Gamma(1+\frac{2}{k})-[\lambda \Gamma(1+\frac{1}{k})]^2\\
         &={\lambda }^2 \Gamma(1+\frac{2}{k})-\lambda^2 [\Gamma(1+\frac{1}{k})]^2\\
         &={\lambda }^2 \Gamma(1+\frac{2}{k})-\lambda^2 \Gamma^2(1+\frac{1}{k})\\
         &={\lambda }^2[ \Gamma(1+\frac{2}{k})- \Gamma^2(1+\frac{1}{k})]\\
\end{align*}

\subsection{One research paper in finance which uses Weibull distribution}



\subsubsection{{Title} \\  \emph{Portfolio value-at-risk with two-sided Weibull distribution: Evidence from cryptocurrency markets} }


\subsubsection{Link to the paper}


Here is a link to the paper: \href{https://www.sciencedirect.com/science/article/pii/S1544612319312024}{https://www.sciencedirect.com/science/article/pii/S1544612319312024}.




\subsubsection{Summary}

The authors of that paper  introduced a new Value-at-Risk(\textbf{VaR}) measurement model based on the two-sided Weibull distribution to assess potential losses in cryptocurrencies.

The Value-at-Risk (\textbf{VaR}) is the minimum loss expected on an investment, over a given time period and at a specific quantile level. It is one of the well-known market risk measures.

For their study, they used dataset on four major cryptocurrencies such as Bitcoin, Litcoin, Ripple and Dash, to compare the performance of this new model with ten other risk models using different evaluation methods. The empirical results showed that the model based on the  Weibull distribution performed better than the other models.



\newpage
\section{CTQ 9}

\newpage
\section{CTQ 10}

   
\end{document}

 