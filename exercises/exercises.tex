\documentclass[11pt]{article}
\usepackage{lipsum} % for dummy text
    \usepackage[breakable]{tcolorbox}
    \usepackage{parskip} % Stop auto-indenting (to mimic markdown behaviour)
    

    \usepackage{float}
    \floatplacement{figure}{H} % forces figures to be placed at the correct location
    \usepackage{xcolor} % Allow colors to be defined
    \usepackage{enumerate} % Needed for markdown enumerations to work
    \usepackage{geometry} % Used to adjust the document margins
    \usepackage{amsmath} % Equations
    \usepackage{amssymb} % Equations
    \usepackage{textcomp} % defines textquotesingle
    % Hack from http://tex.stackexchange.com/a/47451/13684:
    \AtBeginDocument{%
        \def\PYZsq{\textquotesingle}% Upright quotes in Pygmentized code
    }
    \usepackage{upquote} % Upright quotes for verbatim code
    \usepackage{eurosym} % defines \euro

    \usepackage{iftex}
    \ifPDFTeX
        \usepackage[T1]{fontenc}
        \IfFileExists{alphabeta.sty}{
              \usepackage{alphabeta}
          }{
              \usepackage[mathletters]{ucs}
              \usepackage[utf8x]{inputenc}
          }
    \else
        \usepackage{fontspec}
        \usepackage{unicode-math}
    \fi

    \usepackage{fancyvrb} % verbatim replacement that allows latex
    \usepackage[Export]{adjustbox} % Used to constrain images to a maximum size
    \adjustboxset{max size={0.9\linewidth}{0.9\paperheight}}

    % The hyperref package gives us a pdf with properly built
    % internal navigation ('pdf bookmarks' for the table of contents,
    % internal cross-reference links, web links for URLs, etc.)
    \usepackage{hyperref}
    % The default LaTeX title has an obnoxious amount of whitespace. By default,
    % titling removes some of it. It also provides customization options.
    \usepackage{titling}
    \usepackage{longtable} % longtable support required by pandoc >1.10
    \usepackage{booktabs}  % table support for pandoc > 1.12.2
    \usepackage{array}     % table support for pandoc >= 2.11.3
    \usepackage{calc}      % table minipage width calculation for pandoc >= 2.11.1
    \usepackage[inline]{enumitem} % IRkernel/repr support (it uses the enumerate* environment)
    \usepackage[normalem]{ulem} % ulem is needed to support strikethroughs (\sout)
                                % normalem makes italics be italics, not underlines
    \usepackage{mathrsfs}
    

    
 
    % commands and environments needed by pandoc snippets
    % extracted from the output of `pandoc -s`
    \providecommand{\tightlist}{%
      \setlength{\itemsep}{0pt}\setlength{\parskip}{0pt}}
    \DefineVerbatimEnvironment{Highlighting}{Verbatim}{commandchars=\\\{\}}
    % Add ',fontsize=\small' for more characters per line
    \newenvironment{Shaded}{}{}
    \newcommand{\KeywordTok}[1]{\textcolor[rgb]{0.00,0.44,0.13}{\textbf{{#1}}}}
    \newcommand{\DataTypeTok}[1]{\textcolor[rgb]{0.56,0.13,0.00}{{#1}}}
    \newcommand{\DecValTok}[1]{\textcolor[rgb]{0.25,0.63,0.44}{{#1}}}
    \newcommand{\BaseNTok}[1]{\textcolor[rgb]{0.25,0.63,0.44}{{#1}}}
    \newcommand{\FloatTok}[1]{\textcolor[rgb]{0.25,0.63,0.44}{{#1}}}
    \newcommand{\CharTok}[1]{\textcolor[rgb]{0.25,0.44,0.63}{{#1}}}
    \newcommand{\StringTok}[1]{\textcolor[rgb]{0.25,0.44,0.63}{{#1}}}
    \newcommand{\CommentTok}[1]{\textcolor[rgb]{0.38,0.63,0.69}{\textit{{#1}}}}
    \newcommand{\OtherTok}[1]{\textcolor[rgb]{0.00,0.44,0.13}{{#1}}}
    \newcommand{\AlertTok}[1]{\textcolor[rgb]{1.00,0.00,0.00}{\textbf{{#1}}}}
    \newcommand{\FunctionTok}[1]{\textcolor[rgb]{0.02,0.16,0.49}{{#1}}}
    \newcommand{\RegionMarkerTok}[1]{{#1}}
    \newcommand{\ErrorTok}[1]{\textcolor[rgb]{1.00,0.00,0.00}{\textbf{{#1}}}}
    \newcommand{\NormalTok}[1]{{#1}}
    
    % Additional commands for more recent versions of Pandoc
    \newcommand{\ConstantTok}[1]{\textcolor[rgb]{0.53,0.00,0.00}{{#1}}}
    \newcommand{\SpecialCharTok}[1]{\textcolor[rgb]{0.25,0.44,0.63}{{#1}}}
    \newcommand{\VerbatimStringTok}[1]{\textcolor[rgb]{0.25,0.44,0.63}{{#1}}}
    \newcommand{\SpecialStringTok}[1]{\textcolor[rgb]{0.73,0.40,0.53}{{#1}}}
    \newcommand{\ImportTok}[1]{{#1}}
    \newcommand{\DocumentationTok}[1]{\textcolor[rgb]{0.73,0.13,0.13}{\textit{{#1}}}}
    \newcommand{\AnnotationTok}[1]{\textcolor[rgb]{0.38,0.63,0.69}{\textbf{\textit{{#1}}}}}
   
    
    % Define a nice break command that doesn't care if a line doesn't already
    % exist.
    \def\br{\hspace*{\fill} \\* }
    % Math Jax compatibility definitions
    \def\gt{>}
    \def\lt{<}
    \let\Oldtex\TeX
    \let\Oldlatex\LaTeX
    \renewcommand{\TeX}{\textrm{\Oldtex}}
    \renewcommand{\LaTeX}{\textrm{\Oldlatex}}
    % Document parameters
    % Document title
    \title{Advanced Statistical Methods for Modeling and Finance: Solutions for exercises}
\author{Gbètoho Ezéchiel ADEDE}


% \date{\today}
    
    
    
    
    
% Pygments definitions
\makeatletter
\def\PY@reset{\let\PY@it=\relax \let\PY@bf=\relax%
    \let\PY@ul=\relax \let\PY@tc=\relax%
    \let\PY@bc=\relax \let\PY@ff=\relax}
\def\PY@tok#1{\csname PY@tok@#1\endcsname}
\def\PY@toks#1+{\ifx\relax#1\empty\else%
    \PY@tok{#1}\expandafter\PY@toks\fi}
\def\PY@do#1{\PY@bc{\PY@tc{\PY@ul{%
    \PY@it{\PY@bf{\PY@ff{#1}}}}}}}
\def\PY#1#2{\PY@reset\PY@toks#1+\relax+\PY@do{#2}}

\@namedef{PY@tok@w}{\def\PY@tc##1{\textcolor[rgb]{0.73,0.73,0.73}{##1}}}
\@namedef{PY@tok@c}{\let\PY@it=\textit\def\PY@tc##1{\textcolor[rgb]{0.24,0.48,0.48}{##1}}}
\@namedef{PY@tok@cp}{\def\PY@tc##1{\textcolor[rgb]{0.61,0.40,0.00}{##1}}}
\@namedef{PY@tok@k}{\let\PY@bf=\textbf\def\PY@tc##1{\textcolor[rgb]{0.00,0.50,0.00}{##1}}}
\@namedef{PY@tok@kp}{\def\PY@tc##1{\textcolor[rgb]{0.00,0.50,0.00}{##1}}}
\@namedef{PY@tok@kt}{\def\PY@tc##1{\textcolor[rgb]{0.69,0.00,0.25}{##1}}}
\@namedef{PY@tok@o}{\def\PY@tc##1{\textcolor[rgb]{0.40,0.40,0.40}{##1}}}
\@namedef{PY@tok@ow}{\let\PY@bf=\textbf\def\PY@tc##1{\textcolor[rgb]{0.67,0.13,1.00}{##1}}}
\@namedef{PY@tok@nb}{\def\PY@tc##1{\textcolor[rgb]{0.00,0.50,0.00}{##1}}}
\@namedef{PY@tok@nf}{\def\PY@tc##1{\textcolor[rgb]{0.00,0.00,1.00}{##1}}}
\@namedef{PY@tok@nc}{\let\PY@bf=\textbf\def\PY@tc##1{\textcolor[rgb]{0.00,0.00,1.00}{##1}}}
\@namedef{PY@tok@nn}{\let\PY@bf=\textbf\def\PY@tc##1{\textcolor[rgb]{0.00,0.00,1.00}{##1}}}
\@namedef{PY@tok@ne}{\let\PY@bf=\textbf\def\PY@tc##1{\textcolor[rgb]{0.80,0.25,0.22}{##1}}}
\@namedef{PY@tok@nv}{\def\PY@tc##1{\textcolor[rgb]{0.10,0.09,0.49}{##1}}}
\@namedef{PY@tok@no}{\def\PY@tc##1{\textcolor[rgb]{0.53,0.00,0.00}{##1}}}
\@namedef{PY@tok@nl}{\def\PY@tc##1{\textcolor[rgb]{0.46,0.46,0.00}{##1}}}
\@namedef{PY@tok@ni}{\let\PY@bf=\textbf\def\PY@tc##1{\textcolor[rgb]{0.44,0.44,0.44}{##1}}}
\@namedef{PY@tok@na}{\def\PY@tc##1{\textcolor[rgb]{0.41,0.47,0.13}{##1}}}
\@namedef{PY@tok@nt}{\let\PY@bf=\textbf\def\PY@tc##1{\textcolor[rgb]{0.00,0.50,0.00}{##1}}}
\@namedef{PY@tok@nd}{\def\PY@tc##1{\textcolor[rgb]{0.67,0.13,1.00}{##1}}}
\@namedef{PY@tok@s}{\def\PY@tc##1{\textcolor[rgb]{0.73,0.13,0.13}{##1}}}
\@namedef{PY@tok@sd}{\let\PY@it=\textit\def\PY@tc##1{\textcolor[rgb]{0.73,0.13,0.13}{##1}}}
\@namedef{PY@tok@si}{\let\PY@bf=\textbf\def\PY@tc##1{\textcolor[rgb]{0.64,0.35,0.47}{##1}}}
\@namedef{PY@tok@se}{\let\PY@bf=\textbf\def\PY@tc##1{\textcolor[rgb]{0.67,0.36,0.12}{##1}}}
\@namedef{PY@tok@sr}{\def\PY@tc##1{\textcolor[rgb]{0.64,0.35,0.47}{##1}}}
\@namedef{PY@tok@ss}{\def\PY@tc##1{\textcolor[rgb]{0.10,0.09,0.49}{##1}}}
\@namedef{PY@tok@sx}{\def\PY@tc##1{\textcolor[rgb]{0.00,0.50,0.00}{##1}}}
\@namedef{PY@tok@m}{\def\PY@tc##1{\textcolor[rgb]{0.40,0.40,0.40}{##1}}}
\@namedef{PY@tok@gh}{\let\PY@bf=\textbf\def\PY@tc##1{\textcolor[rgb]{0.00,0.00,0.50}{##1}}}
\@namedef{PY@tok@gu}{\let\PY@bf=\textbf\def\PY@tc##1{\textcolor[rgb]{0.50,0.00,0.50}{##1}}}
\@namedef{PY@tok@gd}{\def\PY@tc##1{\textcolor[rgb]{0.63,0.00,0.00}{##1}}}
\@namedef{PY@tok@gi}{\def\PY@tc##1{\textcolor[rgb]{0.00,0.52,0.00}{##1}}}
\@namedef{PY@tok@gr}{\def\PY@tc##1{\textcolor[rgb]{0.89,0.00,0.00}{##1}}}
\@namedef{PY@tok@ge}{\let\PY@it=\textit}
\@namedef{PY@tok@gs}{\let\PY@bf=\textbf}
\@namedef{PY@tok@gp}{\let\PY@bf=\textbf\def\PY@tc##1{\textcolor[rgb]{0.00,0.00,0.50}{##1}}}
\@namedef{PY@tok@go}{\def\PY@tc##1{\textcolor[rgb]{0.44,0.44,0.44}{##1}}}
\@namedef{PY@tok@gt}{\def\PY@tc##1{\textcolor[rgb]{0.00,0.27,0.87}{##1}}}
\@namedef{PY@tok@err}{\def\PY@bc##1{{\setlength{\fboxsep}{\string -\fboxrule}\fcolorbox[rgb]{1.00,0.00,0.00}{1,1,1}{\strut ##1}}}}
\@namedef{PY@tok@kc}{\let\PY@bf=\textbf\def\PY@tc##1{\textcolor[rgb]{0.00,0.50,0.00}{##1}}}
\@namedef{PY@tok@kd}{\let\PY@bf=\textbf\def\PY@tc##1{\textcolor[rgb]{0.00,0.50,0.00}{##1}}}
\@namedef{PY@tok@kn}{\let\PY@bf=\textbf\def\PY@tc##1{\textcolor[rgb]{0.00,0.50,0.00}{##1}}}
\@namedef{PY@tok@kr}{\let\PY@bf=\textbf\def\PY@tc##1{\textcolor[rgb]{0.00,0.50,0.00}{##1}}}
\@namedef{PY@tok@bp}{\def\PY@tc##1{\textcolor[rgb]{0.00,0.50,0.00}{##1}}}
\@namedef{PY@tok@fm}{\def\PY@tc##1{\textcolor[rgb]{0.00,0.00,1.00}{##1}}}
\@namedef{PY@tok@vc}{\def\PY@tc##1{\textcolor[rgb]{0.10,0.09,0.49}{##1}}}
\@namedef{PY@tok@vg}{\def\PY@tc##1{\textcolor[rgb]{0.10,0.09,0.49}{##1}}}
\@namedef{PY@tok@vi}{\def\PY@tc##1{\textcolor[rgb]{0.10,0.09,0.49}{##1}}}
\@namedef{PY@tok@vm}{\def\PY@tc##1{\textcolor[rgb]{0.10,0.09,0.49}{##1}}}
\@namedef{PY@tok@sa}{\def\PY@tc##1{\textcolor[rgb]{0.73,0.13,0.13}{##1}}}
\@namedef{PY@tok@sb}{\def\PY@tc##1{\textcolor[rgb]{0.73,0.13,0.13}{##1}}}
\@namedef{PY@tok@sc}{\def\PY@tc##1{\textcolor[rgb]{0.73,0.13,0.13}{##1}}}
\@namedef{PY@tok@dl}{\def\PY@tc##1{\textcolor[rgb]{0.73,0.13,0.13}{##1}}}
\@namedef{PY@tok@s2}{\def\PY@tc##1{\textcolor[rgb]{0.73,0.13,0.13}{##1}}}
\@namedef{PY@tok@sh}{\def\PY@tc##1{\textcolor[rgb]{0.73,0.13,0.13}{##1}}}
\@namedef{PY@tok@s1}{\def\PY@tc##1{\textcolor[rgb]{0.73,0.13,0.13}{##1}}}
\@namedef{PY@tok@mb}{\def\PY@tc##1{\textcolor[rgb]{0.40,0.40,0.40}{##1}}}
\@namedef{PY@tok@mf}{\def\PY@tc##1{\textcolor[rgb]{0.40,0.40,0.40}{##1}}}
\@namedef{PY@tok@mh}{\def\PY@tc##1{\textcolor[rgb]{0.40,0.40,0.40}{##1}}}
\@namedef{PY@tok@mi}{\def\PY@tc##1{\textcolor[rgb]{0.40,0.40,0.40}{##1}}}
\@namedef{PY@tok@il}{\def\PY@tc##1{\textcolor[rgb]{0.40,0.40,0.40}{##1}}}
\@namedef{PY@tok@mo}{\def\PY@tc##1{\textcolor[rgb]{0.40,0.40,0.40}{##1}}}
\@namedef{PY@tok@ch}{\let\PY@it=\textit\def\PY@tc##1{\textcolor[rgb]{0.24,0.48,0.48}{##1}}}
\@namedef{PY@tok@cm}{\let\PY@it=\textit\def\PY@tc##1{\textcolor[rgb]{0.24,0.48,0.48}{##1}}}
\@namedef{PY@tok@cpf}{\let\PY@it=\textit\def\PY@tc##1{\textcolor[rgb]{0.24,0.48,0.48}{##1}}}
\@namedef{PY@tok@c1}{\let\PY@it=\textit\def\PY@tc##1{\textcolor[rgb]{0.24,0.48,0.48}{##1}}}
\@namedef{PY@tok@cs}{\let\PY@it=\textit\def\PY@tc##1{\textcolor[rgb]{0.24,0.48,0.48}{##1}}}

\def\PYZbs{\char`\\}
\def\PYZus{\char`\_}
\def\PYZob{\char`\{}
\def\PYZcb{\char`\}}
\def\PYZca{\char`\^}
\def\PYZam{\char`\&}
\def\PYZlt{\char`\<}
\def\PYZgt{\char`\>}
\def\PYZsh{\char`\#}
\def\PYZpc{\char`\%}
\def\PYZdl{\char`\$}
\def\PYZhy{\char`\-}
\def\PYZsq{\char`\'}
\def\PYZdq{\char`\"}
\def\PYZti{\char`\~}
% for compatibility with earlier versions
\def\PYZat{@}
\def\PYZlb{[}
\def\PYZrb{]}
\makeatother


    % For linebreaks inside Verbatim environment from package fancyvrb. 
    \makeatletter
        \newbox\Wrappedcontinuationbox 
        \newbox\Wrappedvisiblespacebox 
        \newcommand*\Wrappedvisiblespace {\textcolor{red}{\textvisiblespace}} 
        \newcommand*\Wrappedcontinuationsymbol {\textcolor{red}{\llap{\tiny$\m@th\hookrightarrow$}}} 
        \newcommand*\Wrappedcontinuationindent {3ex } 
        \newcommand*\Wrappedafterbreak {\kern\Wrappedcontinuationindent\copy\Wrappedcontinuationbox} 
        % Take advantage of the already applied Pygments mark-up to insert 
        % potential linebreaks for TeX processing. 
        %        {, <, #, %, $, ' and ": go to next line. 
        %        _, }, ^, &, >, - and ~: stay at end of broken line. 
        % Use of \textquotesingle for straight quote. 
        \newcommand*\Wrappedbreaksatspecials {% 
            \def\PYGZus{\discretionary{\char`\_}{\Wrappedafterbreak}{\char`\_}}% 
            \def\PYGZob{\discretionary{}{\Wrappedafterbreak\char`\{}{\char`\{}}% 
            \def\PYGZcb{\discretionary{\char`\}}{\Wrappedafterbreak}{\char`\}}}% 
            \def\PYGZca{\discretionary{\char`\^}{\Wrappedafterbreak}{\char`\^}}% 
            \def\PYGZam{\discretionary{\char`\&}{\Wrappedafterbreak}{\char`\&}}% 
            \def\PYGZlt{\discretionary{}{\Wrappedafterbreak\char`\<}{\char`\<}}% 
            \def\PYGZgt{\discretionary{\char`\>}{\Wrappedafterbreak}{\char`\>}}% 
            \def\PYGZsh{\discretionary{}{\Wrappedafterbreak\char`\#}{\char`\#}}% 
            \def\PYGZpc{\discretionary{}{\Wrappedafterbreak\char`\%}{\char`\%}}% 
            \def\PYGZdl{\discretionary{}{\Wrappedafterbreak\char`\$}{\char`\$}}% 
            \def\PYGZhy{\discretionary{\char`\-}{\Wrappedafterbreak}{\char`\-}}% 
            \def\PYGZsq{\discretionary{}{\Wrappedafterbreak\textquotesingle}{\textquotesingle}}% 
            \def\PYGZdq{\discretionary{}{\Wrappedafterbreak\char`\"}{\char`\"}}% 
            \def\PYGZti{\discretionary{\char`\~}{\Wrappedafterbreak}{\char`\~}}% 
        } 
        % Some characters . , ; ? ! / are not pygmentized. 
        % This macro makes them "active" and they will insert potential linebreaks 
        \newcommand*\Wrappedbreaksatpunct {% 
            \lccode`\~`\.\lowercase{\def~}{\discretionary{\hbox{\char`\.}}{\Wrappedafterbreak}{\hbox{\char`\.}}}% 
            \lccode`\~`\,\lowercase{\def~}{\discretionary{\hbox{\char`\,}}{\Wrappedafterbreak}{\hbox{\char`\,}}}% 
            \lccode`\~`\;\lowercase{\def~}{\discretionary{\hbox{\char`\;}}{\Wrappedafterbreak}{\hbox{\char`\;}}}% 
            \lccode`\~`\:\lowercase{\def~}{\discretionary{\hbox{\char`\:}}{\Wrappedafterbreak}{\hbox{\char`\:}}}% 
            \lccode`\~`\?\lowercase{\def~}{\discretionary{\hbox{\char`\?}}{\Wrappedafterbreak}{\hbox{\char`\?}}}% 
            \lccode`\~`\!\lowercase{\def~}{\discretionary{\hbox{\char`\!}}{\Wrappedafterbreak}{\hbox{\char`\!}}}% 
            \lccode`\~`\/\lowercase{\def~}{\discretionary{\hbox{\char`\/}}{\Wrappedafterbreak}{\hbox{\char`\/}}}% 
            \catcode`\.\active
            \catcode`\,\active 
            \catcode`\;\active
            \catcode`\:\active
            \catcode`\?\active
            \catcode`\!\active
            \catcode`\/\active 
            \lccode`\~`\~ 	
        }
    \makeatother

    \let\OriginalVerbatim=\Verbatim
    \makeatletter
    \renewcommand{\Verbatim}[1][1]{%
        %\parskip\z@skip
        \sbox\Wrappedcontinuationbox {\Wrappedcontinuationsymbol}%
        \sbox\Wrappedvisiblespacebox {\FV@SetupFont\Wrappedvisiblespace}%
        \def\FancyVerbFormatLine ##1{\hsize\linewidth
            \vtop{\raggedright\hyphenpenalty\z@\exhyphenpenalty\z@
                \doublehyphendemerits\z@\finalhyphendemerits\z@
                \strut ##1\strut}%
        }%
        % If the linebreak is at a space, the latter will be displayed as visible
        % space at end of first line, and a continuation symbol starts next line.
        % Stretch/shrink are however usually zero for typewriter font.
        \def\FV@Space {%
            \nobreak\hskip\z@ plus\fontdimen3\font minus\fontdimen4\font
            \discretionary{\copy\Wrappedvisiblespacebox}{\Wrappedafterbreak}
            {\kern\fontdimen2\font}%
        }%
        
        % Allow breaks at special characters using \PYG... macros.
        \Wrappedbreaksatspecials
        % Breaks at punctuation characters . , ; ? ! and / need catcode=\active 	
        \OriginalVerbatim[#1,codes*=\Wrappedbreaksatpunct]%
    }
    \makeatother
    
    % Exact colors from NB
    \definecolor{incolor}{HTML}{303F9F}
    \definecolor{outcolor}{HTML}{D84315}
    \definecolor{cellborder}{HTML}{CFCFCF}
    \definecolor{cellbackground}{HTML}{F7F7F7}
    
    % prompt
    \makeatletter
    \newcommand{\boxspacing}{\kern\kvtcb@left@rule\kern\kvtcb@boxsep}
    \makeatother
    \newcommand{\prompt}[4]{
        {\ttfamily\llap{{\color{#2}[#3]:\hspace{3pt}#4}}\vspace{-\baselineskip}}
    }
    

  
    

    
    % Prevent overflowing lines due to hard-to-break entities
    \sloppy 
    % Setup hyperref package
    \hypersetup{
      breaklinks=true,  % so long urls are correctly broken across lines
      colorlinks=true,
      urlcolor=urlcolor,
      linkcolor=linkcolor,
      citecolor=citecolor,
      }
    % Slightly bigger margins than the latex defaults
    
    \geometry{verbose,tmargin=1in,bmargin=1in,lmargin=1in,rmargin=1in}

\usepackage{titleps}
\usepackage{fancyhdr}
\usepackage{graphicx}
\pagestyle{myheadings}
\pagestyle{fancy}
\fancyhf{}
\setlength{\headheight}{30pt}
\renewcommand{\headrulewidth}{4pt}
\renewcommand{\footrulewidth}{2pt}
\fancyhead[L]{\includegraphics[width=1cm]{example-image-a}}
\fancyhead[C]{}
\fancyhead[R]{\rightmark}
\fancyfoot[L]{Exercises}
\fancyfoot[C]{Gbètoho Ezéchiel ADEDE}
\fancyfoot[R]{\thepage}

    
    
%*****************************
\begin{document}
    
    \maketitle
    \newpage

    \section{Exercise 1}
  Let us consider an experience of rolling two(balanced) dice.
  
  
  Let the event E such that
\begin{equation*}
E=\{(1,6),(2,5),(3,4),(4,3),(5,2),(6,1)\}
\end{equation*} 
Let us supppose that the dice are cast $N=400$ times and let us find the number of times we should expect to get E occured(at least once).

Let $X$ be the random variable which associates 1 if the event E happens and 0 else. $X  \sim \mathcal B(p=\mathbb{P}(E))$ 

Let Y be the random variable that associates, after N independent rolls of two dice, the number of of trials to get E occured is just N repetitions of the previous Bernoulli experiment, independently and in the same conditions.

Then $Y  \sim \mathcal B(N=400,p=\mathbb{P}(E))$

Let n the number of times we should expect to get the event E occured. Then $n=E(Y)=Np$



\begin{align*}
N=400\\
p
&=\mathbb{P}(E)\\
&=\frac{6}{36}\\
&=\frac{1}{6}\\
\end{align*}

\begin{align*}
n=400\times \frac{1}{6}\
& \approx 66.67\approx 67\\
\end{align*}

\textbf{Conclusion}\\

We should expect to get the event E occured(at least once) after $67$ throws.


\newpage  
    \section{Exercise 2}
   
   Let us consider the ame that consists in flipping a coin twice.
   There are 4 possibles outcomes. There are:
   
   $(T,T)$; $(H,H)$; $(T,H)$ and $(H,T)$
   
   Where $T$ means tails and $H$ means head
   
   \begin{itemize}
   \item[•]Let  $A$ "Both flips got head".
   
   $P(A)=\frac{1}{4}$
   
   \item[•] Let us compute the expected earn after playing $100$ times
   \begin{itemize}
   \item We win $100$ DHS if we get 2 heads(with the probability $\frac{1}{4}$)
   
  \item We lose $60$ DHS ie we win $100 DHS$ if we get 2 tails(with the probability $\frac{1}{4}$)
  \item We win nothing ie $O$ DH otherwise(with the probability $\frac{1}{2}$)
  
  
   
   \end{itemize}
   
   The expected amount to earn for a game if we do not consider the bet is 
  
\begin{align*}
E_1 &=100\times \frac{1}{4}-60\times \frac{1}{4}+0 \times \frac{1}{2}\\
       &=25-15+0\\
       &=10
\end{align*}

By considering the bet, the expected amount to earn for a game is. 

$E_2=E_1-20=10-20=-10$

And then We should expect to lose 10 DHS for just a game.

Let
\begin{align*}
E &=E_1\times 100-20\\
       &=10\times 100-20\\
       &=1000-20\\
       &=980
\end{align*} 

Then after playing the game 100 times, if we suppose that we bet the 20 DHS just once, we will epect to earn 980 DHS.
   
   \end{itemize}
   
   
\newpage 
    \section{Exercise 3}
    
    Let us construct a random variable for exercise 2
    
    Let $\Omega=\{(T,T);(H,H); (T,H); (H,T)\}$

Where $T$ means tails and $H$ means head

Let us consider the random variable Y defined from $\Omega$ to the set of possible earns which is $\{100,-60,0\}$
    
\newpage 
    \section{Exercise 4}
    
    Let $X$ be a random variable such that
    $ P[(a, b)] = b − a, 0 \lt a \lt b \lt 1]$
    
    \subsection{Let us find the PDF $f$ of $X$}
    
    Firstly, let us find the CDF $F$ of $X$
    
    Let $t \in \mathbb{R}$
    
    $F(t)=\mathbb{P}(X\le t)$
    \begin{itemize}
    \item[•] Let $t\le 0$
    
    \begin{align*}
\mathbb{P}(X\le t) &=\mathbb{P}(X\le t \lt 0)\\
       &=\mathbb{P}(X\lt 0)\\
       &=\mathbb{P}(X\in ]-\infty ;0])\\
       &=0\\
\end{align*}

\item[•] Let $t\in ]0;1[$
    
    \begin{align*}
\mathbb{P}(X\le t)
       &=\mathbb{P}(X\in ]0 ;t[)\\
       &=\mathbb{P}[(0 ,t)]\\
       &=t-0\\
       &=t\\
\end{align*}

\item[•] Let $t\in ]1; +\infty[$ 
    
    \begin{align*}
\mathbb{P}(X\le t)
       &=\mathbb{P}(X\in [0 ;1[\cup [1;t[)\\
       &=\mathbb{P}(X\in [0 ;1[)+\mathbb{P}(X\in [1;t[)\\
       &=\mathbb{P}([0 ;1[)+\mathbb{P}([1;t[)\\
       &=1+0\\
       &=1\\
\end{align*} 

    \end{itemize}
    
    
  Therefore
  
  \begin{align*}
F(t) = \begin{cases}
            0 & \text{if } t \le 0\\
            t & \text{if } 0 \lt t \lt 1\\
            1 & \text{if } t \geq 1\\
        \end{cases}
\end{align*}
    
And then 

  \begin{align*}
f(t)=F'(t) &= \begin{cases}
            0 & \text{if } t \le 0\\
            1 & \text{if } 0 \lt t \lt 1\\
            0 & \text{if } t \geq 1\\
        \end{cases}\\
        &=\begin{cases}
        1 & \text{if } 0 \lt t \lt 1\\
            0 & \text{else }\\
        \end{cases}
\end{align*} 
    
\subsection{Let us find $\mathbb{P}([0.2 ;0.4[\cup [0.7;0.85[)$}

\begin{align*}
\mathbb{P}([0.2 ;0.4[\cup [0.7;0.85[)
       &=\mathbb{P}([0.2 ;0.4[)+\mathbb{P}([0.7;0.85[)\\
       &=(0.4-0.2)+(0.85-0.7)\\
       &=0.2+0.15\\
       &=0.2+0.15\\
       &=0.35
\end{align*}
   
\newpage 
    \section{Exercise 5}
    
    Let us suppose a random variable $X$ has it's CDF defined by:\\
    
\begin{align*}
F(x) = \begin{cases}
            0 & \text{if } x \lt 0\\
            \frac{x}{200} & \text{if } 0 \lt x \lt 100\\
            1 & \text{if } x \geq 100\\
        \end{cases}
\end{align*}

\subsection{The graph of F}


\begin{tcolorbox}[breakable, size=fbox, boxrule=1pt, pad at break*=1mm,colback=cellbackground, colframe=cellborder]
\prompt{In}{incolor}{2}{\boxspacing}
\begin{Verbatim}[commandchars=\\\{\}]
\PY{k+kn}{import} \PY{n+nn}{numpy} \PY{k}{as} \PY{n+nn}{np}
\PY{k+kn}{import} \PY{n+nn}{matplotlib}\PY{n+nn}{.}\PY{n+nn}{pyplot} \PY{k}{as} \PY{n+nn}{plt}
\PY{n}{F}\PY{o}{=}\PY{k}{lambda} \PY{n}{x}\PY{p}{:}\PY{n}{np}\PY{o}{.}\PY{n}{where}\PY{p}{(}\PY{l+m+mi}{0}\PY{o}{\PYZlt{}}\PY{n}{x} \PY{o+ow}{and} \PY{n}{x}\PY{o}{\PYZlt{}}\PY{l+m+mi}{100}\PY{p}{,}\PY{n}{x}\PY{o}{/}\PY{l+m+mi}{200}\PY{p}{,}\PY{n}{np}\PY{o}{.}\PY{n}{where}\PY{p}{(}\PY{n}{x}\PY{o}{\PYZlt{}}\PY{l+m+mi}{0} \PY{p}{,}\PY{l+m+mi}{0}\PY{p}{,}\PY{l+m+mi}{1}\PY{p}{)}\PY{p}{)} 
\PY{n}{x}\PY{o}{=}\PY{n}{np}\PY{o}{.}\PY{n}{linspace}\PY{p}{(}\PY{o}{\PYZhy{}}\PY{l+m+mi}{1000}\PY{p}{,}\PY{l+m+mi}{1000}\PY{p}{)}\PY{o}{.}\PY{n}{tolist}\PY{p}{(}\PY{p}{)}
\PY{n}{plt}\PY{o}{.}\PY{n}{plot}\PY{p}{(}\PY{n}{x}\PY{p}{,}\PY{p}{[}\PY{n}{F}\PY{p}{(}\PY{n}{i}\PY{p}{)} \PY{k}{for} \PY{n}{i} \PY{o+ow}{in} \PY{n}{x}\PY{p}{]}\PY{p}{,}\PY{n}{label}\PY{o}{=}\PY{l+s+s1}{\PYZsq{}}\PY{l+s+s1}{F(x)}\PY{l+s+s1}{\PYZsq{}}\PY{p}{,}\PY{n}{color}\PY{o}{=}\PY{l+s+s1}{\PYZsq{}}\PY{l+s+s1}{r}\PY{l+s+s1}{\PYZsq{}}\PY{p}{)}
\PY{n}{plt}\PY{o}{.}\PY{n}{legend}\PY{p}{(}\PY{n}{loc}\PY{o}{=}\PY{l+s+s1}{\PYZsq{}}\PY{l+s+s1}{lower right}\PY{l+s+s1}{\PYZsq{}}\PY{p}{)}
\PY{n}{plt}\PY{o}{.}\PY{n}{show}\PY{p}{(}\PY{p}{)} 
\end{Verbatim}
\end{tcolorbox}

    \begin{center}
    \adjustimage{max size={0.9\linewidth}{0.9\paperheight}}{plotF.png}
    \end{center}
    { \hspace*{\fill} \\}
    
    
\subsection{Let's compute the following probabilities}
\subsubsection*{a)$P(-50\lt X \lt 50)$}
\begin{align*}
P(-50\lt X \lt 50) &=F(50)-F(-50)\\
\end{align*}

\begin{align*}
50\in  \left[0 ; 100\right] \Rightarrow F(50)&=\frac{50}{200}\\
 &=\frac{1}{4}
\end{align*}


\begin{align*}
-50\in  \left]-\infty ; 0\right[ \Rightarrow F(-50)&=0
\end{align*}

Then we have
\begin{align*}
P(-50\lt X \lt 50) &=\frac{1}{4}-0\\
				  &=\frac{1}{4}
\end{align*}

\subsubsection*{b)$P(X=0)$}

\begin{align*}
P(X=0) &=F(0)-\lim_{{x \to 0^-}} F(x)\\	
       &=\frac{0}{200}-\lim_{{x \to 0^-}} 0\\
       &=0
\end{align*}


\subsubsection*{c)$P(X=100)$}

\begin{align*}
P(X=100) &=F(100)-\lim_{{x \to 100^-}} F(x)\\	
       &=1-\lim_{{x \to 100^-}} \frac{x}{200}\\
       &=1-\frac{100}{200}\\
       &=1-\frac{1}{2}\\
       &=\frac{1}{2}
\end{align*}

\newpage 
    \section{Exercise 6}
    Let us suppose a random variable $X$ has it's PDF defined by:\\
    
\begin{align*}
f(x) = \begin{cases}
            kx^2 & \text{if } 0 \leq x \leq 10 \\
            0 & \text{if } x \gt 10.
        \end{cases}
\end{align*}



Let us find $P(7 \lt X \lt 15)$

Let us note $F$ the CDF of $X$

\begin{align*}
P(7 \lt X \lt 15) &= F(15)-F(7) \\
\end{align*}
$\forall x\ge 0 $, we have:
\begin{align*}
F(x) &=\int_\mathbb{-\infty}^x f(t)dt\\
&=\int_{-\infty}^{0} f(t)dt+ \int_0^{x} f(t)dtdt\\
				  &=0+ \int_0^{x} f(t)dt\\
				  &=\int_0^{x} f(t)dt\\
\end{align*}
$t \in \left[0 ; x\right] \iff 0 \le t \le x$

\begin{itemize}
\item[•] If $x \le 10$, then $0 \le t \le x \le 10$ ie $0 \le t \le 10$ and then $f(t)=2kt^2$

So 

\begin{align*}
F(x) &=\int_{0}^x 2kt^2dt\\
&=\frac{2}{3}k\left[t^3\right]_0^x \\
&=\frac{2}{3}kx^3 \\
\end{align*}

\item[•] If $x \ge 10$, then we have:

\begin{align*}
F(x) &=\int_{0}^{10} f(t)dt+\int_{10}^x f(t)dt\\
&=\int_{0}^{10} f(t)dt+0\\
&=\int_{0}^{10} 2kt^2dt \\
&=\frac{2}{3}k\left[t^3\right]_0^{10} \\
&=\frac{2}{3} 10^3 k \\
&=\frac{2000k}{3}
\end{align*}

\end{itemize}

\begin{align*}
F(x) = \frac{2}{3}k\begin{cases}
            x^3 & \text{if } 0 \leq x \leq 10 \\
            1000 & \text{if } x \gt 10.
        \end{cases}
\end{align*}



Let us determine $k$.

\begin{align*}
\int_\mathbb{R} f(t)dt &=\int_\mathbb{-\infty}^0 f(t)dt + \int_0^{10} f(t)dt +\int_{10}^{+\infty} f(t)dt\\
				  &=0+ \int_0^{10} 2kt^2dtdt+0\\
				  &=2k[\frac{1}{3}t^3]_0^{10}\\
				  				  								  &=\frac{2k}{3}[t^3]_0^{10}\\
				  				  								  &=\frac{2k}{3}[t^3]_0^{10}\\
				  				  								  &=\frac{2k}{3}(10^3-0^3)\\
				  				  								  &=\frac{2k}{3}(1000)\\
				  				  								  &=\frac{2000k}{3}\\
\end{align*}

$f$ is a PDF, then $\int_\mathbb{R} f(t)dt=1$

 \begin{align*}
  \int_\mathbb{R} f(t)dt=1 & \Rightarrow \frac{2000k}{3} =1\\ 
 			 &\Rightarrow 2000k=3\\
 			 &\Rightarrow k=\frac{3}{2000}\\  
\end{align*}


And then

\begin{align*}
F(x) &= \frac{2}{3}\times \frac{3}{2000} \begin{cases}
            x^3 & \text{if } 0 \leq x \leq 10 \\
            1000 & \text{if } x \gt 10.
        \end{cases}\\
        &=\frac{1}{1000} \begin{cases}
            x^3 & \text{if } 0 \leq x \leq 10 \\
            1000 & \text{if } x \gt 10.
        \end{cases}\\
         &= \begin{cases}
            \frac{1}{1000}x^3 & \text{if } 0 \leq x \leq 10 \\
            1 & \text{if } x \gt 10.
        \end{cases}\\        
\end{align*}


$15 \gt 10 $ then $F(15)=1$

$7 \in \left[0 ; 10\right] $ then $F(7)=\frac{7^3}{1000}=\frac{343}{1000}$
\begin{align*}
P(7 \lt X \lt 15) &= F(15)-F(7) \\
`                 &=1-\frac{343}{1000}\\ 
 			 	  &=\frac{1000-343}{1000}\\
 			 	  &= 3\times \frac{16}{2000}\\
 			 	  &= \frac{357}{1000}\\  
\end{align*}




    
    
    
    
\newpage 
    \section{Exercise 7}
    Let 
    \begin{itemize}
    \item[•] X be a random variable with f as PDF, such that
      $ f(t)=\frac{1}{\sqrt{2\pi}}\exp(-\frac{1}{2}t^2 ) ; t\in \mathbb{R}$
      \item[•] $Y=X^2$ another random variable.
      \item[•] $F_X$ the CDF of X
      \item[•] $F_Y$ the CDF of Y
      \item[•] $f_Y$ the PDF of Y
    \end{itemize}
    Let us show that $f_Y(y)=\frac{1}{\sqrt{2\pi y}}\exp(-\frac{y}{2}) \forall t\in \mathbb{R}$
   $\forall t\in \mathbb{R}$, we have:
   
\begin{align*}
    F_Y(t) &= P(Y < t) \\
           &= P(X^2 < t) \\
           &= P(Y < \sqrt{t})\\
           &= P(\lvert x \rvert < \sqrt{t}) \text{ since } X^2 \lt t \Rightarrow t \gt 0\\           
           &= P(-\sqrt{t} \lt  \sqrt{t} \lt \sqrt{t} ) \\ 
           &=  F_X(\sqrt{t})-F_X(-\sqrt{t})                    
\end{align*}

\begin{align*}
    f_Y(t) &= F_Y'(t) \\
           &= \frac{1}{2\sqrt{t}}F_X'(\sqrt{t})-(-\frac{1}{2\sqrt{t}})F_X'(\sqrt{t}) \\
          &= \frac{1}{2\sqrt{t}}f(\sqrt{t})-(-\frac{1}{2\sqrt{t}})f(-\sqrt{t}) \\                                          
          &= \frac{1}{2\sqrt{t}}f(\sqrt{t})+\frac{1}{2\sqrt{t}}f(-\sqrt{t}) \\                                          
          &= \frac{1}{2\sqrt{t}}[f(\sqrt{t})+f(-\sqrt{t})] \\
          &= \frac{1}{2\sqrt{t}}(2)(\frac{1}{\sqrt{2\pi}} exp(-\frac{1}{2}t )) \\          
           &= \frac{1}{\sqrt{t}}\frac{1}{\sqrt{2\pi}} exp(-\frac{1}{2}t ) \\ 
           &=\frac{1}{\sqrt{2\pi y}}\exp(-\frac{t}{2})         
\end{align*}

     
\newpage 
    \section{Exercise 8}
Let X be a random variable such that $Var(X)$ exists .


Let's show that 
$Var(X)=E(X^2)-\mu^2$  where $\mu=E(X)$ 

We have by definition:

\begin{align*}
    Var(X) &= E([X-E(X)]^2) \\
           &= E(X^2-2XE(X)+(E(X))^2) \\
           &= E(X^2-2\mu X+\mu^2) \\
           &= E(X^2-2\mu X+\mu^2)\text{ with } \mu=E(X) \\           
           &=  E(X^2)-2\mu E(X)+\mu^2\\      
           &=  E(X^2)-2\mu\mu+\mu^2\text{ because } E(X)=\mu \\  
           &=  E(X^2)-2\mu^2+\mu^2 \\
           &=  E(X^2)-\mu^2 \\                                    
\end{align*}
Hence the result.



\newpage 
    \section{Exercise 9}
    Let X a discrete random variable with pmf f positive at -1, 0,1 and zero elsewhere.
    \subsection{a) Let us find $E(X) \text{ if } p(0)=\frac{1}{4}$}
 \begin{align*}
  E(X^2) &= (-1)^2p(-1)+(0)^2p(0)+(1)^2p(1)+0 \\
       &=p(-1)+p(1) \\       
\end{align*}   

We have

\begin{align*}
  p(-1)+p(0)+p(1)+0 &= p(-1)+p(0)+p(1)\\ 
  					&= p(-1)+\frac{1}{4}+p(1)      \text{ since } p(0)= \frac{1}{4}
\end{align*}
Then 
 
\begin{align*}
  p(-1)+p(1) &= 1-p(0)\\ 
  			&= 1-\frac{1}{4}\\
  			&= \frac{3}{4}\\
\end{align*} 
And then $E(X^2)=\frac{3}{4}$ 

   \subsection{a) Let us find $p(-1)\text{ and } p(1) \text{ if } p(0)=\frac{1}{4} \text{ and if } E(X)= 1$}
   
  \begin{align*}
  E(X) &= (-1)p(-1)+(0)p(0)+(1)p(1)+0 \\
       &=p(1)-p(-1) \\       
\end{align*} 

\begin{align*}
  E(X)= 1 & \Rightarrow p(1)-p(-1)= 1\\
       &=\Rightarrow p(1)=1+p(-1) \text{ } \textcircled{1}      
\end{align*}
$p(-1)+p(0)+p(1)=1$ and $ p(0)=\frac{1}{4}$

Then we get:

\begin{align*}
  p(-1)+p(1) &=  1-\frac{1}{4}\\ 
 			 &= \frac{3}{4}\\    
\end{align*}

And then $p(1)=\frac{3}{4}-P(-1) \textbf{ } \textcircled{2} $

\begin{align*}
  \textcircled{1} = \textcircled{2} & \Rightarrow  1+p(-1)=\frac{3}{4}-p(-1)\\ 
 			 &\Rightarrow 2p(-1)=\frac{3}{4}-1\\
 			 &\Rightarrow 2p(-1)=-\frac{1}{4}\\  
 			 &\Rightarrow p(-1)=-\frac{1}{8} \textbf{(an error?)        }    
\end{align*}
Hence

\begin{align*}
p(-1)&=\frac{3}{4}-p(-1)\\
	 &=\frac{3}{4}+\frac{1}{8}\\
	 &=\frac{24+4}{4\times 8}\\
	 &=\frac{28}{32}\\
	 &=\frac{7}{8}\\
\end{align*}
  
   
\newpage 
    \section{Exercise 10}
   
   Let X a continous random variable with pdf f, such that $f(x)=\frac{1}{3} \text{ for } -1 \lt x \lt 2$
   
   
   Let $M_X$ the moment generating function of $X$
   
   Let $t \in \left]-1 ; 2\right[
 $
  
  \begin{align*}
M_X(t)&=E(e^tX)\\
	 &=\int_\mathbb{R} e^{tx}f(x)dx\\
	 &=\int_{-\infty}^{-1} e^{tx}f(x)dx+\int_{-1}^2 e^{tx}f(x)dx+\int_{2}^{+\infty} e^{tx}f(x)dx\\
	 &=0+\int_{-1}^2 \frac{1}{3}e^{tx}dx+0\\
	 &=\frac{1}{3}\int_{-1}^2 e^{tx}dx\\
\end{align*}
When $t \neq 0$, we have:

 
  \begin{align*}
M_X(t) &=\frac{1}{3}[\frac{1}{t}e^{tx}]_{-1}^2 \\ 
	  &=\frac{1}{3t}[e^{tx}]_{-1}^2 \\
	  &=\frac{1}{3t}(e^{2t}-e^{-t}) \\
\end{align*}
   


 
\newpage 
    \section{Exercise 11}
   Let $X  \sim \mathcal B(n,p)$ and $M_X$ the MGF of $X$
   
   Let us show that $M_X(t)=(1-p+pe^t)^n \text{ } \forall t \in \mathbb{R}$
   
   
   
   
Let $q=1-p$

We have $\mathbb{P}(X=k)=\binom{n}{k}p^k q^{n-k} \forall \text{ } k \in \{0, 1, \ldots, n\} $

Let $t \in \mathbb{R}$
  \begin{align*}
M_X(t) &=E(e^{tX})\\ 
	  &=\Sigma_{i=1}^{n}{ e^{tk} P(Y=k)}\\   
	   &=\Sigma_{i=1}^{n}{ e^{tk} \binom{n}{k}p^k q^{n-k}}\\
	   &=\Sigma_{i=1}^{n}{ p^k (e^{t})^k \binom{n}{k}q^{n-k}}\\
	   &=\Sigma_{i=1}^{n}{(pe^{t})^k \binom{n}{k}q^{n-k}}\\
	   &=\Sigma_{i=1}^{n}{a^k \binom{n}{k}b^{n-k}} \textbf{ with } a =pe^t \text{and }  b=q\\	
	   &=(a+b)^n \text{ from Newten's binomial formula}\\
	   &=(pe^t+1-p)^n \text{ since } a =pe^t \text{ and }  b=q\\
	   &=(1-p+pe^t)^n	  	  
\end{align*}
Hence the result.
   

\newpage 
    \section{Exercise 12}
    
   
   Let X a random variable that has a Poisson ditribution with $\lambda$ as parameter
   
   We have:
   
    $P(X=k)=\frac{\lambda^{k} e^{-\lambda}}{k! } \text{ for k in } \mathbb{N}$
   
   
   Let $M_X$ the moment generating function of $X \textbf{ and }  t \in \mathbb{R}$
  
  \subsection{Let us show that $M_X(t)=e^{\lambda(e^t-1)}$} 
   
  
  \begin{align*}
M_X(t)&=E(e^tX)\\
	&=\Sigma_{k\in \mathbb{N}  }e^{tk}\mathbb{P}(X=k)  \\
	&=\Sigma_{k\in \mathbb{N}  }e^{tk}\frac{\lambda^{k} e^{-\lambda}}{k!}  \\
	&=\Sigma_{k\in \mathbb{N}  }\frac{e^{tk}\lambda^{k} e^{-\lambda}}{k!}  \\
	 &=\Sigma_{k\in \mathbb{N}  }\frac{(e^{t})^k\lambda^{k} e^{-\lambda}}{k!}  \\
	 &=\Sigma_{k\in \mathbb{N}  }\frac{(\lambda e^{t})^k e^{-\lambda}}{k!}  \\
\end{align*}
$\forall x \in \mathbb{R}\text{, } e^x= \Sigma_{n\in \mathbb{N}  }\frac{x^{n}}{n!}$

Then we have:

\begin{align*}
M_X(t)&=e^{-\lambda}e^{\lambda e^t}\\
	&=e^{-\lambda+\lambda e^t}\\
	&=e^{\lambda e^t-\lambda}\\
	&=e^{\lambda( e^t-1)}
\end{align*}
Hence $M_X(t)=e^{\lambda( e^t-1)}$
   
\subsection{Let us find it's mean and variance}

Let $t \in \mathbb{R}$
\begin{itemize}
\item[•]Mean $E(X)$

$E(X)=M_X'(0)$



$M_X(t)=e^{\lambda( e^t-1)}$

\begin{align*}
M_X'(t)&=\lambda e^{t}e^{\lambda( e^t-1)}\\
      &=\lambda e^{t}M_X(t)
\end{align*}
$M_X(0)=e^{e^{0}-1}=e^0=1$

\begin{align*}
M_X'(0)&=\lambda e^{0}M_X(0)\\
	   &=\lambda (1)\\
	   &=\lambda\\
\end{align*}

Then $E(X)=\lambda$

\item[•]Variance $Var(X)$
\end{itemize}
$var(X)=E(X^2)-(E(X))^2$ with $E(X^2)=M_X''(0)$

\begin{align*}
M_X''(t)&=(M_X')'(t)
\end{align*}

\begin{align*}
M_X'(t)&=\lambda e^{t}M_X(t)\\
\end{align*}

\begin{align*}
M_X''(t)&=\lambda[ e^{t}M_X(t)+e^{t}M_X'(t)]\\
&=\lambda e^{t}[ M_X(t)+M_X'(t)]\\
\end{align*}


\begin{align*}
M_X''(0)&=\lambda e^{0}[ M_X(0)+M_X'(0)]\\
        &=\lambda (1)[ 1+\lambda]\\
        &=\lambda (1+\lambda)\\
\end{align*}

Then $E(X^2)=\lambda (1+\lambda)$ and then 

\begin{align*}
Var(X)&=\lambda (1+\lambda)- \lambda ^2\\
        &=\lambda +\lambda ^2 -\lambda ^2]\\
        &=\lambda 
\end{align*}
\newpage 
\section{Exercise 13}
 Let $X  \sim \mathcal G(\alpha,\beta)$ and $M_X$ the MGF of $X$ and $f_X$ it's PDF such that
 $f_X(x)=\frac{1}{\Gamma(a)\times \beta ^\alpha} x^{\alpha-1}e^{-\frac{x}{\beta}}\text{ } \alpha \gt 0 \text{; } \beta \gt 0 \text{; }x\gt 0$
 
 \subsection{Let us show that $M_X(t)=\frac{1}{(1-\beta t)^\alpha}$ if $t\gt \frac{1}{\beta}$}
 
 
\begin{align*}
M_X(t)&=E[e^{tX}]\\
&=\int_{-\infty}^{+\infty} e^{tx}f_X(x)dx\\
&=\int_{0}^{+\infty} e^{tx}\frac{1}{\Gamma(a)\times \beta ^\alpha} x^{\alpha-1}e^{-\frac{x}{\beta}}dx\\
&=\int_{0}^{+\infty}\frac{1}{\Gamma(a)\times \beta ^\alpha} x^{\alpha-1}e^{tx-\frac{x}{\beta}}dx\\
&=\int_{0}^{+\infty}\frac{1}{\Gamma(a)\times \beta ^\alpha} x^{\alpha-1}e^{(t-\frac{1}{\beta})x}dx\\
&=\int_{0}^{+\infty}\frac{1}{\Gamma(a)\times \beta ^\alpha} x^{\alpha-1}e^{-(\frac{1}{\beta}-1)x}dx\\
\end{align*}

Put $y=(\frac{1}{\beta}-1)x$; then 

$\begin{cases}
x=\frac{1}{\frac{1}{\beta}-t}y=\frac{\beta}{1-\beta t}y \\
dx=\frac{1}{\frac{1}{\beta}-t}y=\frac{\beta}{1-\beta t}dy \\
            
\end{cases}\\ $  

And then we get 

\begin{align*}
M_X(t)
&=\int_{0}^{+\infty}\frac{1}{\Gamma(a)\times \beta ^\alpha} x^{\alpha-1}e^{-(\frac{1}{\beta}-1)x}dx\\
&=\int_{0}^{+\infty}\frac{1}{\Gamma(a)\times \beta ^\alpha} {(\frac{\beta}{1-\beta t}y) }^{\alpha-1}e^{-y}\frac{\beta}{1-\beta t}dy\\
&=\int_{0}^{+\infty}\frac{1}{\Gamma(a)\times \beta ^\alpha} {(\frac{\beta}{1-\beta t}) }^{\alpha-1}\frac{\beta}{1-\beta t} y ^{\alpha-1}e^{-y}dy\\
&=\int_{0}^{+\infty}\frac{1}{\Gamma(a)\times \beta ^\alpha} {(\frac{\beta}{1-\beta t}) }^{\alpha} y^{\alpha-1} e^{-y}dy\\
&=\int_{0}^{+\infty}\frac{\beta^\alpha}{\Gamma(a)\times \beta ^\alpha (1-\beta t)^\alpha} y^{\alpha-1} e^{-y}dy\\
&=\int_{0}^{+\infty}\frac{1}{\Gamma(a)\times (1-\beta t)^\alpha} y^{\alpha-1} e^{-y}dy\\
&=\frac{1}{\Gamma(a)\times (1-\beta t)^\alpha}\int_{0}^{+\infty} y^{\alpha-1} e^{-y}dy\\
&=\frac{1}{\Gamma(a)\times (1-\beta t)^\alpha}\Gamma(a) \\
&=\frac{1}{(1-\beta t)^\alpha} \\
\end{align*}    


\subsection{Mean and variance of Gamma distribution}

\begin{itemize}
	\item[•]Mean $E(X)$
	
	
	$E(X)=M_X'(0)$
	
\begin{align*}
M_X(t)  
&=\frac{1}{(1-\beta t)^\alpha} \\
&=(1-\beta t)^{-\alpha} \\	
\end{align*}
	
\begin{align*}
M_X'(t)
&=-\alpha(-\beta)(1-\beta t)^{-\alpha-1} \\
&=\alpha \beta (1-\beta t)^{-\alpha-1} \\	
\end{align*}
	

\begin{align*}
M_X'(0)  
&=\alpha \beta (1-0)^{-\alpha-1} \\
&=\alpha \beta (1)^{-\alpha-1} \\
&=\alpha \beta \times 1 \\
&=\alpha \beta \\
\end{align*}
	
	
	Hence $E(X)=\alpha \beta$
	
	\item[•]Variance $Var(X)$
	
	$Var(X)=E(X^2)-[E(X)]^2$
	
	$E(X^2)=M_X''(0)$
\end{itemize}

\begin{align*}
	M_X'(t)  
	&=\alpha \beta (1-\beta t)^{-\alpha-1} \\
\end{align*}

\begin{align*}
M_X''(t)  
&=\alpha \beta[(-\alpha-1)(-\beta)(1-\beta t)^{-\alpha-1-1}]\\
&=\alpha \beta[\beta(\alpha+1)(1-\beta t)^{-\alpha-2}]\\
&=\alpha \beta ^2(\alpha+1)(1-\beta t)^{-\alpha-2}\\
\end{align*}


\begin{align*}
M_X''(0)  
&=\alpha \beta^2(\alpha+1)(1-0)^{-\alpha-2}\\&=\alpha \beta^2(\alpha+1)(1)^{-\alpha-2}\\
&=\alpha \beta^2(\alpha+1)\times 1\\
&=\alpha \beta^2(\alpha+1)\\
\end{align*}

Hence 

$E(X^2)=\alpha \beta^2(\alpha+1)$

$Var(X)=E(X^2)-[E(X)]^2$

\begin{align*}
Var(X)  
&=E(X^2)-[E(X)]^2\\
&=\alpha \beta^2(\alpha+1)-(\alpha \beta)^2\\
&=\alpha \beta^2(\alpha+1)-\alpha ^2 \beta ^2\\
&= \beta ^2 [\alpha(\alpha+1)-\alpha^2]\\
&= \beta ^2 [\alpha^2 +\alpha-\alpha^2]\\
&= \beta ^2 (\alpha)\\
&=  \alpha \beta ^2\\
\end{align*}


\newpage 

\section{Exercise 14}
 Let $X  \sim \mathcal N(\mu,\sigma^2)$  such that

$\begin{cases}
            P(X\le 89)=0.90 \\
            P(X\le 94)=0.95 \\
\end{cases}$


Let us show that $\mu=71.4$  and that $\sigma^2=189.4$

Let $Z=\frac{X-\mu}{\sigma}$. Then $X=\mu+\sigma Z$ and
\begin{align*}
P(X\le 89)&=P(\mu+\sigma Z\le 89)\\
&=P( Z\le \frac{89-\mu}{\sigma})\\
\end{align*}

And similarly we have
\begin{align*}
P(X\le 94)&=P( Z\le \frac{94-\mu}{\sigma})\\
\end{align*}

We know that $Z  \sim \mathcal N(0,1)$.

Let $\Phi$ the CDF of $Z$. Then we have successively

$\begin{cases}
            P(X\le 89)=0.90=\Phi(\frac{89-\mu}{\sigma}) \\
            P(X\le 94)=0.95=\Phi(\frac{94-\mu}{\sigma} \\
\end{cases}$


$\begin{cases}
     \frac{89-\mu}{\sigma}=\Phi^{-1}(0.90)\textcircled{1}\\
     \frac{90-\mu}{\sigma}=\Phi^{-1}(0.95)\textcircled{2}\\
\end{cases}$

$\textcircled{2}-\textcircled{1}$ gives successively

\begin{align*}
\frac{90-\mu-89+\mu}{\sigma}&=\Phi^{-1}(0.95)-\Phi^{-1}(0.90)\\
\end{align*}

\begin{align*}
\frac{5}{\sigma}&=\Phi^{-1}(0.95)-\Phi^{-1}(0.90)\\
\end{align*}
\begin{align*}
\sigma &=\frac{5}{\Phi^{-1}(0.95)-\Phi^{-1}(0.90)} \\
\end{align*}

The relation $\textcircled{1}$ gives successively

\begin{align*}
89-\mu=\sigma \Phi^{-1}(0.90)\\
\mu=89-\sigma \Phi^{-1}(0.90)\\
\end{align*}

 
    \begin{tcolorbox}[breakable, size=fbox, boxrule=1pt, pad at break*=1mm,colback=cellbackground, colframe=cellborder]
\prompt{In}{incolor}{26}{\boxspacing}
\begin{Verbatim}[commandchars=\\\{\}]
\PY{k+kn}{from} \PY{n+nn}{scipy}\PY{n+nn}{.}\PY{n+nn}{stats} \PY{k+kn}{import} \PY{n}{norm}
\PY{k+kn}{from} \PY{n+nn}{numpy} \PY{k+kn}{import} \PY{n}{power}

\PY{n}{p1} \PY{o}{=} \PY{l+m+mf}{0.90}
\PY{n}{p2}\PY{o}{=}\PY{l+m+mf}{0.95}

\PY{n}{sigma}\PY{o}{=}\PY{l+m+mi}{5}\PY{o}{/}\PY{p}{(}\PY{n}{norm}\PY{o}{.}\PY{n}{ppf}\PY{p}{(}\PY{n}{p2}\PY{p}{)}\PY{o}{\PYZhy{}}\PY{n}{norm}\PY{o}{.}\PY{n}{ppf}\PY{p}{(}\PY{n}{p1}\PY{p}{)}\PY{p}{)}
\PY{n}{sigma\PYZus{}squared}\PY{o}{=}\PY{n}{power}\PY{p}{(}\PY{n}{sigma}\PY{p}{,}\PY{l+m+mi}{2}\PY{p}{)}
\PY{n}{mu}\PY{o}{=}\PY{l+m+mi}{89}\PY{o}{\PYZhy{}}\PY{n}{sigma}\PY{o}{*}\PY{n}{norm}\PY{o}{.}\PY{n}{ppf}\PY{p}{(}\PY{l+m+mf}{0.90}\PY{p}{)}

\PY{n+nb}{print}\PY{p}{(}\PY{l+s+s1}{\PYZsq{}}\PY{l+s+s1}{sigma squared:}\PY{l+s+s1}{\PYZsq{}}\PY{p}{,}\PY{n}{sigma\PYZus{}squared}\PY{p}{)}
\PY{n+nb}{print}\PY{p}{(}\PY{l+s+s1}{\PYZsq{}}\PY{l+s+s1}{mu:}\PY{l+s+s1}{\PYZsq{}}\PY{p}{,}\PY{n}{mu}\PY{p}{)}
\end{Verbatim}
\end{tcolorbox}

    \begin{Verbatim}[commandchars=\\\{\}]
sigma squared: 189.4106020419289
mu: 71.36245122609756
    \end{Verbatim}

    \begin{tcolorbox}[breakable, size=fbox, boxrule=1pt, pad at break*=1mm,colback=cellbackground, colframe=cellborder]
\prompt{In}{incolor}{ }{\boxspacing}

\end{tcolorbox}

Hence we have approximatively

$\mu=71.4$  and $\sigma^2=189.4$






   
   




    % Add a bibliography block to the postdoc
    
    
    
\end{document}
